The \Gls{galapagos} Instruction Set Architecture is the Instruction Set Archtecture designed for the \Gls{barricelli} computer for this project. The ISA is loosely based on the well-known and tested MIPS architecture\cn, but borrows inspiration from many other different sources as well. Especially, inspirational is the the core design principles used when designing \emph{MIPS}.

\begin{description}
  \item[Design principle 1] Simplicity favours regularity~\cite[p.~79]{compOrgDes}. 
  \item[Design principle 2] Smaller is faster ~\cite[p.~81]{compOrgDes}
  \item[Design principle 3] Make the common case fast.~\cite[p.~86]{compOrgDes}
\end{description}

\todo{Fix these references}

\todo{Link galapagos ISA to these principles}

All of these principles visible in both the ISA and the data path of the galapagos architecture. As with MIPS, galapagos rely on relatively few instructions classes. The instructions formats are constructed to be regular, in order to make the decode process simpler. The different information in the instruction is always located on the same positions.  


The galapagos ISA is in fact an RISC architecture. The instructions are kept simple and are only performing very specific and small tasks. That is, the instructions are low level instructions working executed directly on the hardware without need for additional decoding in form of microinstructions. The different instruction formats supported by the galapagos architecture can be seen in figure(?) \todo{list instructions formats} \todo{clarify, RISC vs CISC}. 





The Galapagos ISA was designed and fully specified quite early in the project, which made it an important resource for the rest of the component design process.

The ISA is thoroughly documented in appendix \vref{appendix:isa}.

One of the requirements\cn was that the ISA should support genetic-specific instructions to facilitate performant genetic algorithms programming.
Present in the \Gls{galapagos} ISA are the genetic instructions \texttt{ldg}, \texttt{stg} and \texttt{setg}.
They are the instructions for loading and storing \glspl{individual} to the genetics accelerator, and configuring the genetics accelerator, respectively.
Refer to the ISA documentation in appendix \vref{appendix:isa} for in-depth documentation about what they do.

