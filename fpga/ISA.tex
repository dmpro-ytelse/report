The \Gls{galapagos} Instruction Set Architecture is the instruction set archtecture designed for the \Gls{barricelli} computer for this project.
The architecture is loosely based on the well-known and tested \gls{MIPS} architecture\cn, but borrows inspiration from many other different sources as well.
Especially inspirational for the design of the \Gls{galapagos} ISA have been the \gls{MIPS} core design principles, which can be found in table \vref{fpga:tbl:mips-design-principles}.
This was chosen in accordance with the core design principles.

\begin{table}[H]
\centering
    \begin{tabular}{l l} 
     \textbf{Design principle 1} & Simplicity favours regularity.~\cite[p.~79]{compOrgDes}. \\
     \textbf{Design principle 2} & Smaller is faster.~\cite[p.~81]{compOrgDes} \\
     \textbf{Design principle 3} & Make the common case fast.~\cite[p.~86]{compOrgDes} \\
    \hline
\end{tabular}
    \caption{MIPS Design Principles}
    \label{fpga:tbl:mips-design-principles}
\end{table}

The Galapagos ISA was designed and fully specified quite early in the project, which made it an important resource for the rest of the component design process.

While the ISA is thoroughly documented in appendix \vref{appendix:isa}, the rest of this section will present a short overview for the reader's convenience.



\todo{Fix these references}

\todo{Link galapagos ISA to these principles}

The \Gls{galapagos} instruction set architecture is a RISC architecture.
The instructions are kept simple, and only perform very specific and small tasks.
That is, the instructions are low-level instructions executed directly in hardware without the need for additional decoding in form of microinstructions or the like.

\subsection{Instruction Formats}

As with \Gls{MIPS}, \Gls{galapagos}, in true \gls{RISC} fashion, has relatively few instruction formats.
These instruction formats are constructed to be regular, which implies that the different information types contained in an instruction are always located in the same positions, when possible.
This makes the instruction decoding process in the processor much simpler.
This is done in accordance with design principle 1 of table \vref{fpga:tbl:mips-design-principles}.

The three instruction formats used in \Gls{galapagos} are the RRR, RRI and RI formats.
They are named after the types of data they contain.
RRR contains three register addresses.
RRI contains two register addresses and one immediate.
RI contains one register address and a larger immediate.

Every instruction has a set of conditional flags that may be set.
Though these flags, a programmer can decide whether or not an instruction can be executed.
This allows for branchless conditional execution of single instructions.
These conditional signals allow for many clever applications - a \gls{nop} instruction can be implemented as any instruction with the condition set to ``never execute''.
Indeed, even conditional branching is implemented as a branch-less conditional!
Again, for a more detailed docummentation of the \gls{galapagos} instruction set architecture, the reader is encouraged to read appendix \vref{appendix:isa}.


\subsection{Genetics Instructions}

One of the requirements\cn was that the ISA should support genetic-specific instructions to facilitate performant genetic algorithms programming.
Present in the \Gls{galapagos} ISA are the genetic instructions \texttt{ldg}, \texttt{stg} and \texttt{setg}.
They are the instructions for loading and storing \glspl{individual} to the genetics accelerator, and configuring the genetics accelerator, respectively.
With these instructions available to the programmer, using the genetics accelerator is easy and painless.
The reader may refer to the ISA documentation in appendix \vref{appendix:isa} for in-depth documentation about how the genetics instructions are used.

