A key component in any genetic algorithm is a decent source of (pseudo-)random numbers.
Genetic algorithms require diversity in the individuals to prevent reaching a local maximum, as discussed in section (?) \todo{add me}. 
To achieve this, the architecture need to support a way to produce random numbers for the genetic pipeline.   

The Barricelli computer has a hardware pseudo-random number generator module built into its genetics accelerator.

When designing a pseudo-random number generator, there is always a trade-off between generating ``good'' random numbers, and generating numbers them quickly.
Having high performance as a design goal\cn, it was desirable to design a pseudo-random number generator that is as fast as possible while still bein usable for genetic algorithms.

The pseudo-random number generator uses a linear shift feedback-based taps algorithm to efficiently generate random numbers.

The design is inspired by \todo{cite this: http://www.ece.ualberta.ca/~elliott/ee552/studentAppNotes/1999f/Drivers\_Ed/lfsr.html}.

This algorithm is shown in algorithm \vref{algorithm:prng}.

\begin{algorithm}[H]
\SetAlgoLined
\DontPrintSemicolon
\KwData{\\
    $ LFSR = $ a 32-bit linear shift feedback register containing the previous random number \\
    $ taps = "01000110000000000000000100000000" $
}
\KwResult{A new random number}
\Begin{
    $ feedback \longleftarrow LFSR[31] $\;
    \For{i from 31 to 0, non-inclusive}{
        \If{taps[i - 1] = 1}{
            $ LFSR[i] \longleftarrow LFSR[i - 1] \oplus feedback $\;
        }\Else{
            $ LFSR[i] \longleftarrow LFSR[i - 1] $\;
        }
    }
    $ LFSR[0] = feedback $\;
    \Return{LFSR}
}
\caption{Pseudo-random number generation algorithm}
\label{algorithm:prng}
\end{algorithm}

