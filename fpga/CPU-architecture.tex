\begin{figure}[H]
\includegraphics[width=\textwidth]{fpga/fig/processor_architecture.png}
\caption{Processor Architecture}
\label{figure:fpga-architecture}
\end{figure}

\

Figure \ref{figure:fpga-architecture} shows the final processor architecture. Most of the figures seen in this picture as for instance "fitness core", are abstractions of more complex logic at lower levels (mostly MSI and LSI components). 
The processor architecture designed for the Barricelli computer is a very clean design, and the key to its high performance lies in its simplicity.
The architecture contains a number of general cores, which in this context are named fitness cores.
The fitness cores are general purpose cores in the sense that they are programmable and Turing complete, but for genetic algorithm applications the cores are intended to calculate fitness scores of individuals.

The number of fitness cores is configurable.
The reference implementation of the Barricelli computer is configured to have 7 fitness cores.

Common genetic algorithms operations are performed by a separate hardware accelerator pipeline.
This accelerator consists of several operation-specific special cores for selection, crossover and mutation.
The fitness cores and the genetic pipeline are all connected to a single data bus.
To avoid any memory synchronization issues the data bus is controlled by a central arbitration unit.

The processor architecture is illustrated in figure \vref{figure:fpga-architecture}.


\subsection{Instruction Memory}
\label{subsec:fpga-instruction-memory}
The Barricelli is a Harvard machine.
The memory is split into instruction and data memory.
This is done to achieve better memory throughput, because both memories can be accessed simultaneously.
The instruction memory is constructed as an memory hierarchy, with slower external memory (SRAM) and faster, internal on-chip caches (BRAM).
This separation enables a high instruction thoughput by using fast memory without compromising the data storage capabilities of a larger, slower chip.
Each fitness core has its own private instruction memory which buffers instructions to decrease the number of slow memory accesses needed during runtime.

Access to an instruction cache is handled by its dedicated cache controller responsible for locating and transferring instructions from the instruction memory.
In case of a cache miss the data requesting core is halted until the instruction is transferred from memory.
A pseudo-algorithm describing the cache-fetch operation can be found in algorithm \vref{algorithm:cahe-operation}.
This scheme is created to resolve the conflicts that arise from using shared memory. 

The internal resources on the FPGA are a lot faster to access.
Because of this the architecture tries to use the caches as much as possible.

\begin{algorithm}[H]
\SetAlgoLined
\DontPrintSemicolon
\KwData{
    $ a = $ an instruction address \\
    $ Ci = $ an array of instructions \\
    $ Ca = $ an array of the corresponding addresses \\
    $ M = $ the instruction memory, indexable by instruction addresses
}
\KwResult{The instruction at address $ a $}
\Begin{
    \If{$ a = Ca[A \bmod{512}] $}{
        \Return{$ Ci[A \bmod{512}] $}
    }\Else{
        $ Caa \bmod{512}] \longleftarrow a $\;
        $ Ci[a \bmod{512}] \longleftarrow M[a] $\;
        \Return{$ Ci[A \bmod{512}] $}
    }
}
\caption{Fetching an instruction from the cache}
\label{algorithm:cache-operation}
\end{algorithm}


\subsection{Data Memory}
\label{subsec:fpga-data-memory}
The \gls{galapagos} architecture is a \gls{MIMD} architecture with shared memory.
In the \Gls{barricelli} computer, a central memory controller is responsible for synchronizing memory access on the shared data bus.
Each component that wants to access memory must go through the memory controller, and follow the proper memory access request protocol.
The controller is constructed in a way that only allows one fitness core to be able to carry out a memory request at a single time.
In case of multiple memory requests, the controller performs a selection deciding in which order the requesting cores is granted the bus.
The precise technique of selection can be seen in algorithm \ref{algorithm:round-robin-selection}.
This may introduce a potential bottleneck for memory-bound problems.
For this reason, each fitness core has a generous 31 general purpose registers, which should reduce the data memory load quite a bit.

\begin{figure}[H]
\begin{algorithm}[H]
\SetAlgoLined
\DontPrintSemicolon
\KwData{$ Requests = $ requests signals from the fitness cores\newline 
$ Request = $ 2-bits specifying the operation}
\Begin{
    $ Requests \longleftarrow $ requests from the fitness cores\;
    \While{$ True $}{
        \For {request in Requests} {
            \If{request $=$ asserted}{
                performMemoryOperation()
            }
        }
        
    }
}
\caption{Round-robin selection}
\label{algorithm:round-robin-selection}
\end{algorithm}
\end{figure}


The selection algorithm is based on round-robin scheduling.
The request signals of \emph{fitness cores} are checked in turn to check if one of the cores has requested the memory bus.
The type of request is determined by combination of two request signals sent by each \emph{fitness core}.
The signals refer to either a \emph{NOP}, \emph{READ}, or \emph{WRITE} operation.
In case of \emph{NOP}, the algorithm moves on to check the next state request lines.
It continues doing this in this fashion until a \emph{READ} or \emph{WRITE} request is encountered. 

When a \emph{READ} or \emph{WRITE} operation is encountered, the \emph{data controller} starts to carry out the request from the fitness core.
Performing a memory operation takes at least four cycles, as the processor word size is 64 bits, while the memory bus to the external memory chips are only 16 bits wide.
Because of this, data needs to be shuffled across the bus 16-bits at a time, which accounts for the four cycle minimum for data operations.

For the external memory to be operated correctly by the memory controller, the proper control signals need to be set at the correct times. The signals required differs depending on the type of operation, \emph{READ} or \emph{WRITE}. The timing diagrams can be seen in figure \ref{fpga:fig:timing:dmem:read} and \ref{fpga:fig:timing:dmem:write}, respectively. 



\begin{figure}[H]
  \centering
  \includegraphics[width=\textwidth]{fpga/fig/timing/data_mem_read.pdf}
  \caption{Data memory read cycle}
  \label{fpga:fig:timing:dmem:read}
\end{figure}

\begin{figure}[H]
  \centering
  \includegraphics[width=\textwidth]{fpga/fig/timing/data_mem_write.pdf}
  \caption{Data memory write cycle}
  \label{fpga:fig:timing:dmem:write}
\end{figure}

As is immediately apparent in figures \vref{fpga:fig:timing:dmem:read} and \vref{fpga:fig:timing:dmem:write}, the number of cycles required for load and store operations are are 5 and 13 cycles, respectively.
A state machine is implemented in the \emph{data memory controller} to handle interfacing with the external memory chips.
This state machine is responsible for controlling that the different signals are set according to the diagrams.
For more detailed view of the \emph{Data memory controller}, the reader is advised to study the state machine diagram in figure \ref{fpga:fig:mem:data_memory_ctrl_state_machine} and the accompanying data path in figure \ref{fpga:fig:mem:data_memory_ctrl}.

\begin{figure}[H]
  \centering
  \includegraphics[width=\textwidth]{fpga/fig/memory_ctrl_state_machine.png}
  \caption{Data memory controller state machine}
  \label{fpga:fig:mem:data_memory_ctrl_state_machine}
\end{figure}


\begin{figure}[H]
  \centering
  \includegraphics[width=\textwidth]{fpga/fig/memory_ctrl.png}
  \caption{Data memory controller signals mapping}
  \label{fpga:fig:mem:data_memory_ctrl}
\end{figure}



\subsection{Rated and Unrated Pools}
In the beginning and the end of the genetics accelerator pipeline lies the rated and the unrated pools, respectively.
The rated and unrated pools are caches of genetics individuals that waiting to either 1) get selected and be sent through the pipeline, 2) die, or 3) get picked up by a fitness core for fitness ranking.
The rated pool contains individuals stored with a fitness score, and are the indivuiduals that have just been inserted into the accelerator by a fitness core.
The unrated pool contains individuals that have no fitness score calculated.
They are the new individuals produced by the accelerator pipeline, and are waiting to be picked up and rated by a fitness core.

The rated and unrated pools are implemented in \gls{BRAM} on the FPGA for as fast access times as possible.
This is essential to achieve a high memory throughput when executing the algorithms.

It is important to note that the two pools are designed as separate \gls{BRAM} devices.
This is done to achieve even better memory throughput.
The increased throughput is achieved because the different computational units can work on the rated and unrated pools simultaneously.
For instance while one fitness core is storing a ranked individual, another fitness core may be fetching a new individual for ranking. 

Access to the \gls{rated pool} and the \gls{unrated pool} is managed by control units referred to as the \gls{rated controller} and the \gls{unrated controller}.
As with the \gls{data controller} for data memory access, these controllers are responsible for granting access to the rated and unrated pools.
As shown in figure \vref{fpga:fig:genetic:genetic_pipeline}, the genetics accelerator has its own data buses connected to the fitness cores separate from the data bus that is connected to the regular shared data memory, to increase performance by reducing bus conflicts.

The controllers are based on the same principles as the \gls{data controller} described in section \vref{subsec:fpga-data-memory}.
When in need of performing genetic operations, the fitness cores need to request the data bus by setting some request signals.
The combination of these signals refer to the operation the fitness core requests from the genetic controller. 

The controllers continuously perform simple round-robin request handling schemes in order to grant bus access to the next requesting fitness cores, and to the genetic pipeline.
The logic for the \gls{rated controller} is implemented as a state machine, while the \gls{unrated controller}'s simple structure allows it not have one at all.

As the timing diagrams in figure \vref{fpga:fig:timing:genetic:rated_genetic_proc} and \vref{fpga:fig:timing:genetic:unrated_genetic_proc} show, both controllers are highly optimized for speed. The single occation where the bus is unused is the cycle in which the \ref{genetic controller} tells the \gls{rated controller} that it no longer requires access.

\begin{figure}[H]
  \centering
  \includegraphics[width=\textwidth]{fpga/fig/timing/rated_genetic_proc.pdf}
  \caption{Fitness values and individuals being read by the selection cores followed by a fitness core writing a new fitness and individual.}
  \label{fpga:fig:timing:genetic:rated_genetic_proc}
\end{figure}

\begin{figure}[H]
  \centering
  \includegraphics[width=\textwidth]{fpga/fig/timing/unrated_genetic_proc.pdf}
  \caption{A two-induvidual selection core access }
  \label{fpga:fig:timing:genetic:unrated_genetic_proc}
\end{figure}



\subsection{PRNG Module}
A key component in any genetic algorithm worth its salt is a decent source of (pseudo-)random numbers.
Genetic algorithms require diversity in the individuals to prevent reaching a local maximum, as discussed in section (?) \todo{add me}. 
To achieve this, the architecture need to support a way to produce random numbers for the genetic pipeline.   

The Barricelli computer has a hardware pseudo-random number generator module built into its genetics accelerator.

When designing a pseudo-random number generator, there is always a trade-off between generating ``good'' random numbers, and generating them fast.
Having high performance as a design goal\cn, it was desirable to design a pseudo-random number generator that is as fast as possible while still meeting the minimum requirements for randomness that is needed for successfully using it in a genetics algorithm application.

The pseudo-random number generator uses a linear shift feedback-based taps algorithm to efficiently generate random numbers.
This algorithm is shown in algorithm \vref{algorithm:prng}.

\begin{algorithm}[H]
\SetAlgoLined
\DontPrintSemicolon
\KwData{\\
    $ LFSR = $ a 32-bit linear shift feedback register containing the previous random number \\
    $ taps = "01000110000000000000000100000000" $
}
\KwResult{A new random number}
\Begin{
    $ feedback \longleftarrow LFSR[31] $\;
    \For{i from 31 to 0, non-inclusive}{
        \If{taps[i - 1] = 1}{
            $ LFSR[i] \longleftarrow LFSR[i - 1] \oplus feedback $\;
        }\Else{
            $ LFSR[i] \longleftarrow LFSR[i - 1] $\;
        }
    }
    $ LFSR[0] = feedback $\;
    \Return{LFSR}
}
\caption{Pseudo-random number generation algorithm}
\label{algorithm:prng}
\end{algorithm}
    
 

\subsection{Fitness Core} \label{fpga:fitness:ss:design_of_the_fitness_core}
    \subsection{Design of fitness core} \label{fpga:fitness:ss:design_of_the_fitness_core}


\subsubsection{Data Path} \label{fpga:fitness:sss:data_path}
The design of the fitness core is highly influenced by MIPS. The core is designed as a five stage pipeline. The contents of the different stages in the pipeline, however, differs from the original MIPS architecture. The group decided to create a slightly different ISA that combines ideas from both MIPS and ARM. The idea was to make it as simple as possible, and at the same time harvest efficiency by instruction level parallelism. More advanced features like branch prediction and instruction scheduling are not taken into consideration while designing the data path. 

Hazard and branching schemes are made simple. Hazards are resolved with the \emph{forwarding unit}, which forwards data if dependencies are detected. Branching are resolved with use of \emph{conditional} codes in each instructions. These codes specifies wether the instruction should be execute or not. 



\subsubsection{Control Unit} \label{fpga:fitness:sss:control_unit}
In order to execute different instructions classes in the data path there is need for a control unit. The responsibility of the control unit is to configure all the different components for the current CPU operation so that the desired computation will emerge from the flow of data. The control unit achieves this by setting the control signals of the relevant control signals of the relevant components to the values associated with the current instruction. Note that different instructions classes requires different use of the data path.

The control unit sets up the components based on the \emph{FUNCTION CODE} and the \emph{OPERATION CODE} of the instruction.     



\subsubsection{Conditionals} \label{fpga:fitness:sss:conditionals}
Like ARM , the galapagos architecture use conditional codes in order to determine if an instruction should be executed or not. Instead of using explicit branch instructions every instruction carries a 4-bit conditional code. 


\subsubsection{The ALU}\label{fpga:fitness:sss:the_alu}

\subsubsection{Forwarding Unit} \label{fpga:fitness:sss:forwarding_unit}

\subsubsection{Memory Controller} \label{fpga:fitness:sss:memory_controller}

\subsubsection{Genetic Pool Controller} \label{fpga:fitness:sss:genetic_pool_controller}

\subsubsection {Case Study} \label{fpga:fitness:sss:case_study}




\fxnote {The hazard scheme may change if time}
 \label{fpga:subsection:fitness_core}


\subsection{The Genetic Pipeline}
\label{fpga:subsection:genetic_pipeline}
\todo{some words about the genetics accelerator}
The galapagos architecture includes a highly specialised pipeline for performing genetic operations. The pipeline is based on the observation that selection, crossover, and mutation works similar for a specific subset of problems. These can therefore be implemented as hardware accelerators constructed for performing one specific task. Constructing such accelerators has been proven to be very beneficial regarding performance. Designing specialised hardware is usually simpler and thereby more effective than constructing general purpose components.\todo{Bullshit ?} This pipeline will effectively relieve the general cores, the fitness cores, from computing the evolution of individuals. The idea is that these will make the fitness cores able to only focus on the computation of fitness ranking, which is considered computational intensive. In the mean time the \emph{genetic pipeline} can produce new data for ranking. These operations could have been performed by the processor, however, the processor is badly suited for these kind of operations. Note that the instructions in the pipeline actually uses 5 cycles in order to complete propagate through the pipeline. It is a far better to only use one cycle in order to complete the one specific operation.  

The genetic pipeline is constructed with three specialised cores for performing selection, crossover, and mutation. These are operations that occurs frequently in genetic algorithms. These are connected to two internal memory banks on the \emph{FPGA}, namely the unrated and rated pool.


-Abstraction for the programmer. Simpler to program.
-Do not need components like ALU
- effective 
- Less control over the genetic pipeline
- 



\subsubsection {Selection Core} \label{fpga:selection:ss:selection_core}
    \todo{some words about the genetics accelerator}
The Selection Core is the first of the cores in the genetics accelerator.
It is responsible for selecting apropriate individuals from the rated pool population in the genetics pipeline and pass them on to the other cores so that they may operate on them.

The selection core is designed based on a tournament selection algorithm.
Tournament selection is a selection scheme that aims to quickly find an individual with a high score from an unsorted list in a way that does not guarantee that the selected individual is the one with the highest score.
These goals are healthy goals for a selection algorithm to have when used in a genetic algorithm.

The tournament selection-based selection algorithm is precisely described in algorithm \vref{algorithm:tournament-selection}.
In laymans terms, it is designed to select a single individual from a random position in the rated pool.
The current best and the random selected is compared to each other with use of a comparator.
The best chromosome is stored and used in the next tournament round.
After some number of tournaments the current best is transferred to the crossover core.
The selection core is responsible for letting the rest of the genetic pipeline know when it can fetch the next chromosome. 

\begin{figure}[H]
\begin{algorithm}[H]
\SetAlgoLined
\DontPrintSemicolon
\KwData{$ P = $ A pool of rated individuals, $ r = $ number of rounds in tornament (configurable, $ 0 \le r \le 31 $)}
\KwResult{A selected individual}
\Begin{
    $ k \longleftarrow 0 $\;
    $ bestIndividual \longleftarrow $ random individual from P\;
    \While{$ k < r $}{
        $ individual \longleftarrow $ random individual from P\;
        \If{ $ fitness(individual) > fitness(bestIndividual) $ }{
            $ bestIndividual \longleftarrow individual $\;
        }
        $ k \longleftarrow k + 1 $\;
    }

    \Return{$ bestIndividual $}\;
}
\caption{The tournament selection used in the selection core.}
\label{algorithm:tournament-selection}
\end{algorithm}
\end{figure}

The selection core is designed with efficiency in mind.
The overall time spent in the genetic pipeline must be smaller than the time spent ranking the chromosomes.
Note that the fitness cores are connected to the same memory bus as the genetic pipeline.
This could potentially lead to a memory bottleneck resulting in starvation.
The selection core tries to overcome this fact by reducing the memory access to a minimum.
Note that the selection core has reserved the memory bus during the ongoing tournament.
This implies that port used by the selection core is unavailable to others during this time.
It is designed to not use the memory more than it absolutely have to.
For instance, if the current fitness value is greater than the fitness value just fetched.
The selection core will not bother fetching the accompanying chromosome.
Ensuring that the memory resources are not wasted.
This is accomplished with an \emph{state machine}. 
\todo{show the state machine in a diagram}



\subsubsection{Data Path}

\begin{figure}

  \centering
  % Trim er [left bottom right top]
  \includegraphics[trim=5cm 20cm 1cm 1cm, clip=true ]{fpga/fig/data_path_selection_core.pdf}
  \caption{Architecture block diagram}
  \label{fpga:fig:selection:selection_core_data_path}
\end{figure}





\todo{this}





\subsubsection{Control Unit} \label{fpga:selection:sss:control_unit}



\begin{figure}

  \centering
  % Trim er [left bottom right top]
  \includegraphics[trim=5cm 16cm 1cm 5cm, clip=true ]{fpga/fig/selection_core_state_machine.pdf}
  \caption{Architecture block diagram}
  \label{fpga:fig:selection:selection_core_data_path}
\end{figure}



\subsubsection {Comparator} \label{fpga:selection:sss:comparator}

\todo{relevant ?}





 \label{fpga:subsection:selection_core}

\subsubsection{Crossover Core} \label{fpga:crossover:ss:crossover_core}
    TITLE: Crossover

Crossover is the second phase in the genetics accelerator. Two inputs are forwarded from the two selections cores as "parents", and two outputs are the "children" of the inputs, containing bits from both parents. All the bits from both the parents are forwarded in the children, but at some points the bit-patterns are switched on the children, based on random_number inputs from the PRNG. Henceforth this is called crossover.

There are three distinct crossover functions that are implented: Crossover_split, crossover_doublesplit and crossover_xor.

SUB_TITLE: Crossover_split 

FIGURE Y1: Crossover_split

The first function, crossover_split, performs crossover from a selected bit number in the children and until the edge (bit number 0). This can be seen in figure Y1. The values in the parents are represented with X's and Y's, and a single X or Y can have the value 0 or 1, independent of each pther.
The bit_number for starting crossover is based on the value of a 6-bit input random_number, which is provided by the PRNG. This value ranges from 0 to 63.

(Describe how it technically works?)

SUB_TITLE: Crossover_doublesplit

FIGURE Y2: Crossover_doublesplit

The second function, crossover_doublesplit, is similar to the crossover_split-function, but in additionally to having a starting bit for crossover, it also has an ending bit where the crossover starts, instead of reaching the edge at bit nr. 0. PRNG provides with 2 6-bit inputs, random_number1 and random_number2, whose values selects the starting bit and the ending bit for the crossover. These values range from 0 to 63, and if both are the same, then only one bit will be selected for crossover.

(Describe how it technically works?)

SUB_TITLE: Crossover_xor

FIGURE Y3:  Crossover_xor

The third function, crossover_xor, performs crossover bit by bit, based on the 64-bit input random_number. For each bit number i that has the value 1, the function will perform crossover on the children on the same bit number i. This function is called XOR because of use of XOR-gates in earlier version of the function, and the principle is still the same in the current version: For each bit number i in the child, the value will be from bit number i from one and only one parent.

(Describe how it technically works?)

SUB_TITLE: Crossover_toplevel

FIGURE Y4: Crossover_toplevel.png (Who should generate this? Per Thomas, so that architecture style/image is similar to rest?)

Crossover is implemented on the genetics accelerator as a toplevel containing 3 cores, one for each function, as well as a fourth path with no crossover. In addition to the two parent inputs and 64-bit input random_number, the toplevel also takes in a control_number used for determining which crossover function is to be used: Split, doublesplit, xor, "party mode" or no crossover at all. Party mode is choosing crossover function at random, based on the 2 LS bits in the random_number. In this way, whenever inputs are sent through the crossover_toplevel, different functions may be used at different times. These are the control values:
* 000 - Split
* 001 - Doublesplit
* 010 - XOR
* 011 - No crossover
* 1XX - Party mode, in which case these are the random control values:
    * 00 - Split
    * 01 - Doublesplit
    * 10 - XOR
    * 11 - No crossover \label{fpga:subsection:crossover_core}

\subsubsection{Mutation Core}\label{fpga:mutation:ss:mutation_core}
    TITLE: Mutation

Mutation is the final phase in the genetics accelerator. This is implemented as cores that takes in a forwarded child as input and may perform mutaion on selected bits before passing on the result. 

FIGURE ZXT: Mutation function

The mutation core takes in the 64-bit child as input, as well as a 32-bit random\_number and a 6-bit chance\_input as inputs. As it can be seen in the example in figure ZXT, all bits that are not mutated are represented by an A, and mutated bits are represented with M. The values in each A or M can be 0 or 1, independent of each other. The value M at bit number i is the opposite of the original value A at same bit number i in the input.
The 6 first bits in the random\_number is compared to the chance\_input, and mutation happens only if the value of these bits are less than the chance input. For each different value in chance\_input, the user may increase or decrease the chance of mutation by about $0.015 (1 / 2^64)$. If the chance input is set to 000000, no mutation will ever happen, and the user may in this way deny disable the mutation core.

The next two bits in the random number (bits nr. 25-24) are used to determine how many mutations will happen. There are 4 different values, therefore there can be 1-4 mutations.
The next 24 bits are used to determine which bits are to be mutated. 6 bits are used for finding each bit number. This is similar to what is done in the split and doublesplit functions in the crossover phase. These values are numbered, representing their bit field:
- Nr. 1: 5-0
- Nr. 2: 11-6
- Nr. 3: 17-12
- Nr. 4: 23-18
These are numbered after the amount of allowed mutation. Nr. 1 will always happen when a mutation occurs, while nr. 4 happens only when the amount\_number allows for 4 mutations.

Note that if more than one of these numbers point to the same bit number, the output M will still be the inverted from the original input. For instance, if numbers 1 and 2 (bits 11-6 and 5-0) have the value 000110, and therefore point at bit number 6, the same mutation will still happen as if only one of these numbers were 000110. If the input bit was 1, the mutated will be 0, and vice versa.
In the example provided in figure ZXT, the 6 first bits of the random\_number are less than the chance\_input, therefore a mutation happens. Bits 23-0 have the values 30, 14, 23 and 5. Because the value of bits 25-24, the mutation\_amount, is 10, there are 3 mutations, and the fourth does not occur (though the figure shows where it would have occured if allowed). \label{fpga:subsection:mutation_core}



