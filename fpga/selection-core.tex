The selection core is designed based on a tournament selection algorithm.
It is designed to select an chromosome from a random position in the rated pool.
The current best and the random selected is compared to each other with use of an comparator.
The best chromosome is stored and used in the next tournament round.
After some number of tournaments the current best is transferred to the crossover core.
The selection core is actually responsible for letting the rest of the genetic pipeline know when it can fetch the next chromosome. 

The selection core is designed with efficiency in mind.
The overall time spent in the genetic pipeline must be smaller than the time spent ranking the chromosomes.
Note that the fitness cores are connected to the same memory bus as the genetic pipeline.
This could potentially lead to a memory bottleneck resulting in starvation.
The selection core tries to overcome this fact by reducing the memory access to a minimum.
Note that the selection core has reserved the memory bus during the ongoing tournament.
This implies that port used by the selection core is unavailable to others during this time.
It is designed to not use the memory more than it absolutely have to.
For instance, if the current fitness value is greater than the fitness value just fetched.
The selection core will not bother fetching the accompanying chromosome.
Ensuring that the memory resources are not wasted.
This is accomplished with an \emph{state machine}. 
\todo{show the state machine in a diagram}



\subsubsection{Data Path}
Upon the beginning of the data path design, the group wanted to determine the the components required to perform the selection.
In this specific case the group decided to design a data path able to perform a tournament selection.
The resulting architecture is made as simple as possible.
It is composed of \emph{flip flops}, \emph{control unit} and an \emph{comparator}.
The different components are connected as seen in figure.

As seen in figure (?), the selection core requires that some control signals are set in order to run the different operations.
Note that the procedures for handling chromosomes and fitness values are completely different.
In order for the data path to be able to carrie out these various operations, there is need for some control unit to synchronise different actions in the data path. 



\fxnote{Add figure of selection core}


\subsubsection{Control Unit} \label{fpga:selection:sss:control_unit}



\subsubsection {Comparator} \label{fpga:selection:sss:comparator}



\subsubsection{Case study} \label{fpga:selection:sss:case_study}



