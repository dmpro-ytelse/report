The \Gls{barricelli} is a \Gls{harvard machine}.
Because of this the memory is split into instruction and data memory. 
This is done to achieve better memory throughput, because both memories can be accessed simultaneously.
The instruction memory is organized in a two layer memory hierarchy, with slower external memory (\gls{SRAM}) and faster, internal on-chip caches (\gls{BRAM}).
This separation combines the high instruction throughput of fast on-board memory with the comfortably spacious data storage capabilities of a larger, slower chip external chip.

Each fitness core has its own private instruction memory cache which buffers instructions to decrease the number of slow memory accesses needed during run-time.
Access to an instruction cache is handled by a fitness core's dedicated cache controller, which is responsible for locating and transferring instructions from the instruction memory.
In case of a cache miss, the data-requesting core is halted until the instruction is transferred from memory.
A pseudo-algorithm describing the cache fetch operation can be found in algorithm \vref{algorithm:cache-operation}.
This scheme is created to resolve the conflicts that arise from using shared memory. 

\begin{figure}[H]
\begin{algorithm}[H]
\SetAlgoLined
\DontPrintSemicolon
\KwData{ $ a = $ an instruction address \newline
$ Ci = $ an array of instructions \newline
$ Ca = $ an array of the corresponding addresses \newline
$ M = $ the instruction memory, indexable by instruction addresses
}
\KwResult{The instruction at address $ a $}
\Begin{
    \If{$ a = Ca[A \bmod{512}] $}{
        \Return{$ Ci[A \bmod{512}] $}
    }\Else{
        $ Caa \bmod{512}] \longleftarrow a $\;
        $ Ci[a \bmod{512}] \longleftarrow M[a] $\;
        \Return{$ Ci[A \bmod{512}] $}
    }
}
\caption{Fetching an instruction from the cache}
\label{algorithm:cache-operation}
\end{algorithm}
\end{figure}
