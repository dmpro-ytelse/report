\todo{some words about the genetics accelerator}
The galapagos architecture includes a highly specialised pipeline for performing genetic operations. The pipeline is based on the observation that selection, crossover, and mutation works similar for a specific subset of problems. These can therefore be implemented as hardware accelerators constructed for performing one specific task. Constructing such accelerators has been proven to be very beneficial regarding performance. Designing specialised hardware is usually simpler and thereby more effective than constructing general purpose components.\todo{Bullshit ?} This pipeline will effectively relieve the general cores, the fitness cores, from computing the evolution of individuals. The idea is that these will make the fitness cores able to only focus on the computation of fitness ranking, which is considered computational intensive. In the mean time the \emph{genetic pipeline} can produce new data for ranking. These operations could have been performed by the processor, however, the processor is badly suited for these kind of operations. Note that the instructions in the pipeline actually uses 5 cycles in order to complete propagate through the pipeline. It is a far better to only use one cycle in order to complete the one specific operation.  

The genetic pipeline is constructed with three specialised cores for performing selection, crossover, and mutation. These are operations that occurs frequently in genetic algorithms. These are connected to two internal memory banks on the \emph{FPGA}, namely the unrated and rated pool.


-Abstraction for the programmer. Simpler to program.
-Do not need components like ALU
- effective 
- Less control over the genetic pipeline
- 



\subsubsection {Selection Core} \label{fpga:selection:ss:selection_core}
    \todo{some words about the genetics accelerator}
The Selection Core is the first of the cores in the genetics accelerator.
It is responsible for selecting apropriate individuals from the rated pool population in the genetics pipeline and pass them on to the other cores so that they may operate on them.

The selection core is designed based on a tournament selection algorithm.
Tournament selection is a selection scheme that aims to quickly find an individual with a high score from an unsorted list in a way that does not guarantee that the selected individual is the one with the highest score.
These goals are healthy goals for a selection algorithm to have when used in a genetic algorithm.

The tournament selection-based selection algorithm is precisely described in algorithm \vref{algorithm:tournament-selection}.
In laymans terms, it is designed to select a single individual from a random position in the rated pool.
The current best and the random selected is compared to each other with use of a comparator.
The best chromosome is stored and used in the next tournament round.
After some number of tournaments the current best is transferred to the crossover core.
The selection core is responsible for letting the rest of the genetic pipeline know when it can fetch the next chromosome. 

\begin{figure}[H]
\begin{algorithm}[H]
\SetAlgoLined
\DontPrintSemicolon
\KwData{$ P = $ A pool of rated individuals, $ r = $ number of rounds in tornament (configurable, $ 0 \le r \le 31 $)}
\KwResult{A selected individual}
\Begin{
    $ k \longleftarrow 0 $\;
    $ bestIndividual \longleftarrow $ random individual from P\;
    \While{$ k < r $}{
        $ individual \longleftarrow $ random individual from P\;
        \If{ $ fitness(individual) > fitness(bestIndividual) $ }{
            $ bestIndividual \longleftarrow individual $\;
        }
        $ k \longleftarrow k + 1 $\;
    }

    \Return{$ bestIndividual $}\;
}
\caption{The tournament selection used in the selection core.}
\label{algorithm:tournament-selection}
\end{algorithm}
\end{figure}

The selection core is designed with efficiency in mind.
The overall time spent in the genetic pipeline must be smaller than the time spent ranking the chromosomes.
Note that the fitness cores are connected to the same memory bus as the genetic pipeline.
This could potentially lead to a memory bottleneck resulting in starvation.
The selection core tries to overcome this fact by reducing the memory access to a minimum.
Note that the selection core has reserved the memory bus during the ongoing tournament.
This implies that port used by the selection core is unavailable to others during this time.
It is designed to not use the memory more than it absolutely have to.
For instance, if the current fitness value is greater than the fitness value just fetched.
The selection core will not bother fetching the accompanying chromosome.
Ensuring that the memory resources are not wasted.
This is accomplished with an \emph{state machine}. 
\todo{show the state machine in a diagram}



\subsubsection{Data Path}

\begin{figure}

  \centering
  % Trim er [left bottom right top]
  \includegraphics[trim=5cm 20cm 1cm 1cm, clip=true ]{fpga/fig/data_path_selection_core.pdf}
  \caption{Architecture block diagram}
  \label{fpga:fig:selection:selection_core_data_path}
\end{figure}





\todo{this}





\subsubsection{Control Unit} \label{fpga:selection:sss:control_unit}



\begin{figure}

  \centering
  % Trim er [left bottom right top]
  \includegraphics[trim=5cm 16cm 1cm 5cm, clip=true ]{fpga/fig/selection_core_state_machine.pdf}
  \caption{Architecture block diagram}
  \label{fpga:fig:selection:selection_core_data_path}
\end{figure}



\subsubsection {Comparator} \label{fpga:selection:sss:comparator}

\todo{relevant ?}





 \label{fpga:subsection:selection_core}

\subsubsection{Crossover Core} \label{fpga:crossover:ss:crossover_core}
    TITLE: Crossover

Crossover is the second phase in the genetics accelerator. Two inputs are forwarded from the two selections cores as "parents", and two outputs are the "children" of the inputs, containing bits from both parents. All the bits from both the parents are forwarded in the children, but at some points the bit-patterns are switched on the children, based on random_number inputs from the PRNG. Henceforth this is called crossover.

There are three distinct crossover functions that are implented: Crossover_split, crossover_doublesplit and crossover_xor.

SUB_TITLE: Crossover_split 

FIGURE Y1: Crossover_split

The first function, crossover_split, performs crossover from a selected bit number in the children and until the edge (bit number 0). This can be seen in figure Y1. The values in the parents are represented with X's and Y's, and a single X or Y can have the value 0 or 1, independent of each pther.
The bit_number for starting crossover is based on the value of a 6-bit input random_number, which is provided by the PRNG. This value ranges from 0 to 63.

(Describe how it technically works?)

SUB_TITLE: Crossover_doublesplit

FIGURE Y2: Crossover_doublesplit

The second function, crossover_doublesplit, is similar to the crossover_split-function, but in additionally to having a starting bit for crossover, it also has an ending bit where the crossover starts, instead of reaching the edge at bit nr. 0. PRNG provides with 2 6-bit inputs, random_number1 and random_number2, whose values selects the starting bit and the ending bit for the crossover. These values range from 0 to 63, and if both are the same, then only one bit will be selected for crossover.

(Describe how it technically works?)

SUB_TITLE: Crossover_xor

FIGURE Y3:  Crossover_xor

The third function, crossover_xor, performs crossover bit by bit, based on the 64-bit input random_number. For each bit number i that has the value 1, the function will perform crossover on the children on the same bit number i. This function is called XOR because of use of XOR-gates in earlier version of the function, and the principle is still the same in the current version: For each bit number i in the child, the value will be from bit number i from one and only one parent.

(Describe how it technically works?)

SUB_TITLE: Crossover_toplevel

FIGURE Y4: Crossover_toplevel.png (Who should generate this? Per Thomas, so that architecture style/image is similar to rest?)

Crossover is implemented on the genetics accelerator as a toplevel containing 3 cores, one for each function, as well as a fourth path with no crossover. In addition to the two parent inputs and 64-bit input random_number, the toplevel also takes in a control_number used for determining which crossover function is to be used: Split, doublesplit, xor, "party mode" or no crossover at all. Party mode is choosing crossover function at random, based on the 2 LS bits in the random_number. In this way, whenever inputs are sent through the crossover_toplevel, different functions may be used at different times. These are the control values:
* 000 - Split
* 001 - Doublesplit
* 010 - XOR
* 011 - No crossover
* 1XX - Party mode, in which case these are the random control values:
    * 00 - Split
    * 01 - Doublesplit
    * 10 - XOR
    * 11 - No crossover \label{fpga:subsection:crossover_core}

\subsubsection{Mutation Core}\label{fpga:mutation:ss:mutation_core}
    TITLE: Mutation

Mutation is the final phase in the genetics accelerator. This is implemented as cores that takes in a forwarded child as input and may perform mutaion on selected bits before passing on the result. 

FIGURE ZXT: Mutation function

The mutation core takes in the 64-bit child as input, as well as a 32-bit random\_number and a 6-bit chance\_input as inputs. As it can be seen in the example in figure ZXT, all bits that are not mutated are represented by an A, and mutated bits are represented with M. The values in each A or M can be 0 or 1, independent of each other. The value M at bit number i is the opposite of the original value A at same bit number i in the input.
The 6 first bits in the random\_number is compared to the chance\_input, and mutation happens only if the value of these bits are less than the chance input. For each different value in chance\_input, the user may increase or decrease the chance of mutation by about $0.015 (1 / 2^64)$. If the chance input is set to 000000, no mutation will ever happen, and the user may in this way deny disable the mutation core.

The next two bits in the random number (bits nr. 25-24) are used to determine how many mutations will happen. There are 4 different values, therefore there can be 1-4 mutations.
The next 24 bits are used to determine which bits are to be mutated. 6 bits are used for finding each bit number. This is similar to what is done in the split and doublesplit functions in the crossover phase. These values are numbered, representing their bit field:
- Nr. 1: 5-0
- Nr. 2: 11-6
- Nr. 3: 17-12
- Nr. 4: 23-18
These are numbered after the amount of allowed mutation. Nr. 1 will always happen when a mutation occurs, while nr. 4 happens only when the amount\_number allows for 4 mutations.

Note that if more than one of these numbers point to the same bit number, the output M will still be the inverted from the original input. For instance, if numbers 1 and 2 (bits 11-6 and 5-0) have the value 000110, and therefore point at bit number 6, the same mutation will still happen as if only one of these numbers were 000110. If the input bit was 1, the mutated will be 0, and vice versa.
In the example provided in figure ZXT, the 6 first bits of the random\_number are less than the chance\_input, therefore a mutation happens. Bits 23-0 have the values 30, 14, 23 and 5. Because the value of bits 25-24, the mutation\_amount, is 10, there are 3 mutations, and the fourth does not occur (though the figure shows where it would have occured if allowed). \label{fpga:subsection:mutation_core}


