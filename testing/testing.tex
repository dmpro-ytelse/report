When designing a computer with a custom architecture from scratch, it is important to continually test and evaluate the correctness of the solution at all possible stages, to ensure that final product is a success.
This section documents and explaines the rationale behind the different types of tests that have been performed.

\section{Testing the Processor}

Baricelli's processor has been tested at four different levels: \gls{VHDL}-based unit test simulations of the different subcomponents,  \gls{VHDL}-based integration test simulations of each processing unit, \gls{VHDL}-based system test simulations of the entire system interfacing against a mock SCU and mock memory, and finally physical integration tests of the processor programmed onto the FPGA of the Barricelli.

\todo{what about timing simulations?}

\subsection{\gls{VHDL}-based Subcomponent Unit Test Simulations}

Unit testing VHDL entites is extremely important in a large and complex design like the Barricelli.
For this project, almost every component, perhaps except the most trivial entities, is tested in an automated or semi-automated VHDL test bench.
A tool was developed to ease the automation of VHDL test running and validation, modeled after the leading test runners in the software industry, such as JUnit\cn and Karma\cn.
This tool enabled tests to be written using easy-to-use self-evaluating tests that compare signals at specific times against expected values.

The goal of these unit tests is to ensure that the building block components work as expected when reacting to specified input.

Results of these tests can be found in \todo{we need to stash results somewhere}.

\subsection{\gls{VHDL}-based Processing Unit Integration Test Simulations}

Each processing unit, which each consists of several interconnected subcomponents, has been simulated for integration testing.
The goal of these tests are to verify that the different subcomponents interface correctly with eachother, and that the behaviour of the supercomponent is as expected.

\subsubsection{Testing the Fitness Core}

\todo{about tb\_fitness\_core.vhd}

\subsubsection{Testing the Genetics Pipeline}

\todo{about tb\_genetics\_pipeline.vhd}

\subsection{\gls{VHDL}-based System test Simulations}
\label{section:testing:fpga:system-tests}

The toplevel simulation test bench of the barricelli computer, which simulates the entire FPGA as a black box interfacing against the external components, supports pre-loading entire programs into a mocked instruction memory component.
The \Gls{galapagos assembler} supports outputting assembled programs compiled to one of these mock memory components, meaning that testing new programs in a simulated environment is an easy and fun process.

\todo{the title of the test should be above the table in which its results are displayed, not just as the caption (rendered below) of the table}

A formal description of the system tests performed at this level can be found in tables
\ref{testing:fitness:pipeline_test},
\ref{testing:fitness:branch_taken},
\ref{testing:fitness:branch_not_taken},
\ref{testing:fitness:conditional_taken},
\ref{testing:fitness:conditional_not_taken},
\ref{testing:fitness:load_data},
\ref{testing:fitness:store_data},
\ref{testing:fitness:store_gene},
and
\ref{testing:fitness:load_gene}.

\begin{table}[H]
  \begin{tabular}{r | p{8cm}}
    \noalign{\smallskip}\hline\noalign{\smallskip}
    
    What to test:  & Observe that RRI and RRR instructions propagate correctly through the pipeline, 
                     and produce the correct result.\\

    \noalign{\smallskip}\hline\noalign{\smallskip}

    How to test:  & The program in listing \todo{create listing}, consisting of both RRR and RRI instructions,
                    are loaded into memory with the test framework. The execution of the instructions are observed with
                    isim.\\

    \noalign{\smallskip}\hline\noalign{\smallskip}

    Pass criteria: & The flow of data is according to the architecture presented in figure. \todo{add reference}
                   Register 1, 2, 3 are loaded with 9, 10 and 19, respectively.   \\
    
     \noalign{\smallskip}\hline\noalign{\smallskip}

    Results: &  What are the result of the test. \\
   \noalign{\smallskip}\hline\noalign{\smallskip}
  
  
  \end{tabular}
  \caption{RRR and RRI instructions}
  \label{testing:fitness:pipeline_test}
\end{table}


\begin{table}[H]
  \begin{tabular}{r | p{8cm}}
    \noalign{\smallskip}\hline\noalign{\smallskip}
    
    What to test:  & Check if the branch address is calculated correctly, and an conditional
                     jump is performed to this address. \\

    \noalign{\smallskip}\hline\noalign{\smallskip}

    How to test:   &  The program in listing \todo{create listing}, is loaded into a test bench.
                      This simple program consists of a simple loop performing some arithmetic
                      operations that store values to registers. The execution of the
                      program is simulated with isim to verify the result \\

    \noalign{\smallskip}\hline\noalign{\smallskip}

    Pass criteria: &  The branch is taken. The instructions located in the $fetch stage$, 
                       $decode stage$, and $execute stage$ are flushed. The results in registers
                       should be X, X, and X in registers X, X and X, respectively. \\

    \noalign{\smallskip}\hline\noalign{\smallskip}
    
    Results: &  Registers X, X, and X is contained in the registers. The instructions in $fetch
                stage$, $decode stage$ and $execute stage$ does not perform any  
                changes to the register file. \\
   \noalign{\smallskip}\hline\noalign{\smallskip}
  
  
  
  \end{tabular}
  \caption{Branch taken}
  \label{testing:fitness:branch_taken}
\end{table}

\begin{table}[H]
  \begin{tabular}{r | p{8cm}}
    \noalign{\smallskip}\hline\noalign{\smallskip}
    
    What to test:  & Check if the conditional jump is disregarded when performing conditional
                     that always evaluate to false.\\

    \noalign{\smallskip}\hline\noalign{\smallskip}

    How to test:   &  The program in listing \todo{create listing}, is loaded into a test bench. 
                       The simple program consists of conditionals that evaluate to false. The
                       execution of the program is simulated with isim, and the results are
                       verified. \\

    \noalign{\smallskip}\hline\noalign{\smallskip}

    Pass criteria: & Execution of the program should not store any data to the registers.\\

    \noalign{\smallskip}\hline\noalign{\smallskip}
    
    Results: &   No data is stored to registers. \\
   \noalign{\smallskip}\hline\noalign{\smallskip}
  
  
  
  \end{tabular}
  \caption{Branch not taken}
  \label{testing:fitness:branch_not_taken}
\end{table}

\begin{table}[H]
  \begin{tabular}{r | p{8cm}}
    \noalign{\smallskip}\hline\noalign{\smallskip}
    
    What to test:  & Check if conditional instruction are executed when they
                     always are evaluated to true.   \\

    \noalign{\smallskip}\hline\noalign{\smallskip}

    How to test:  & The program in listing \todo{create listing}, is loaded into test bench. The 
                    simple program consists of a set with simple conditional instructions that
                    always evaluate to true. The execution of the program is simulated with isim, 
                    and the result is verified. 
    \\

    \noalign{\smallskip}\hline\noalign{\smallskip}

    Pass criteria: & The instructions propagates normally through the pipeline. The different instruction are executed and their results are written to the register file. \\

    \noalign{\smallskip}\hline\noalign{\smallskip}
    
    Results: &  \\
   \noalign{\smallskip}\hline\noalign{\smallskip}
  
  
  
  \end{tabular}
  \caption{Conditional instruction executed }
  \label{testing:fitness:conditional_taken}
\end{table}

\begin{table}[H]
  \begin{tabular}{r | p{8cm}}
    \noalign{\smallskip}\hline\noalign{\smallskip}
    
    What to test:  & Check if conditional instructions are executed when they always evaluate to 
                     false. \\

    \noalign{\smallskip}\hline\noalign{\smallskip}

    How to test:   &  The program in listing \ref{testing:listing:conditional-not_executed}, is loaded into a test bench. 
                       The simple program consists of a set of simple conditional instructions that         
                       always evaluate to false. The execution of the program is observed in 
                       ISim,and the results are verified. The content of register r1 is observed.\\

    \noalign{\smallskip}\hline\noalign{\smallskip}

    Pass criteria: & The second instruction, the conditional \emph{ADDI}, is not executed. The content of register r1 is 1.  \\

    \noalign{\smallskip}\hline\noalign{\smallskip}
    
    Results: & The content of register r1 is 1. The conditional \emph{ADDI} instruction is not executed. .  \\
   \noalign{\smallskip}\hline\noalign{\smallskip}
  
  
  
  \end{tabular}
  \caption{Conditional instruction not executed}
  \label{testing:fitness:conditional_not_taken}
\end{table}

\begin{table}[H]
  \begin{tabular}{r | p{8cm}}
    \noalign{\smallskip}\hline\noalign{\smallskip}
    
    What to test:  & Observe that LOAD instructions is able to read from memory, and load the
                     memory content into the specified registers.  \\

    \noalign{\smallskip}\hline\noalign{\smallskip}

    How to test:   & The program in listing \todo{create listing}, consisting mainly of LOAD
                     instructions. These are loaded into a testbench, and simulated with 
                     isim. \\
                     

    \noalign{\smallskip}\hline\noalign{\smallskip}

    Pass criteria: &  \\

    \noalign{\smallskip}\hline\noalign{\smallskip}
    
    Results: &  \\
   \noalign{\smallskip}\hline\noalign{\smallskip}
  
  
  
  \end{tabular}
  \caption{Load data}
  \label{testing:fitness:load_data}
\end{table}

\begin{table}[H]
  \begin{tabular}{r | p{8cm}}
    \noalign{\smallskip}\hline\noalign{\smallskip}
    
    What to test:  & \\

    \noalign{\smallskip}\hline\noalign{\smallskip}

    How to test:   & \\

    \noalign{\smallskip}\hline\noalign{\smallskip}

    Pass criteria: & \\

    \noalign{\smallskip}\hline\noalign{\smallskip}
    
    Results: &  \\
   \noalign{\smallskip}\hline\noalign{\smallskip}
  
  
  
  \end{tabular}
  \caption{Store data}
  \label{testing:fitness:pipeline_test}
\end{table}
\begin{table}[H]
  \begin{tabular}{r | p{8cm}}
    \noalign{\smallskip}\hline\noalign{\smallskip}
    
    What to test:  &  \\

    \noalign{\smallskip}\hline\noalign{\smallskip}

    How to test:   &  \\

    \noalign{\smallskip}\hline\noalign{\smallskip}

    Pass criteria: &  \\

    \noalign{\smallskip}\hline\noalign{\smallskip}
    
    Results: &  \\
   \noalign{\smallskip}\hline\noalign{\smallskip}
  
  
  
  \end{tabular}
  \caption{Store gene}
  \label{testing:fitness:store_gene}
\end{table}

\begin{table}[H]
  \begin{tabular}{r | p{8cm}}
    \noalign{\smallskip}\hline\noalign{\smallskip}
    
    What to test:  &  Observe that a gene is fetched from the unrated code, and stored in the
                      specified register\\

    \noalign{\smallskip}\hline\noalign{\smallskip}

    How to test:   &  The program in listing \ref{testing:listing:load-gene}, consisting of LOAD GENE
                      instructions. These are loaded into a test bench and simulated with 
                      ISim. The content of the location of the distributed counters are checked against the data 
                      loaded to the fitness cores.  \\

    \noalign{\smallskip}\hline\noalign{\smallskip}

    Pass criteria: &  The data fetched from the rated pool is the same gene transmitted to the
                      fitness core. \\

    \noalign{\smallskip}\hline\noalign{\smallskip}
    
    Results: &  Success \\
   \noalign{\smallskip}\hline\noalign{\smallskip}
  
  
  
  \end{tabular}
  \caption{Load gene}
  \label{testing:fitness:load_gene}
\end{table}

\begin{table}[H]
\center
\begin{tabular}{|l | c | c | c | c |c|}
    \hline
    Clock cycle & IF & ID & EX & MEM & WB \\
    \hline
    3 & ADD 3, 1, 2  & ADDI 2, 10 & ADDI 1, 10   &              &              \\
    4 &              & ADD 3, 1, 2  & ADDI 2, 10    & ADDI 1, 10   &              \\
    5 &              &              & ADD 3, 1, 2   & ADDI 2, 10   & ADDI 1, 10    \\
    6 &              &              &               & ADD 3, 1, 2  & ADDI 2, 10    \\
    7 &              &              &               &              & ADD 3, 1, 2  \\
    \hline
\end{tabular}
\caption{Instruction flow}
\label{testing:tbl:instrflow}
\end{table}





\subsection{Physical Integration Tests}

Finally, on the physical board, the processor was tested by running the same programs as in the system tests described in section \vref{section:testing:fpga:system-tests}.
These programs were programmed into the instruction memory of the processor by the SCU.

\section{Testing the PCB}
During and after the components were soldered on the PCB board, the board were tested to ensure that the powergrid were working as it was supposed to.
For the first test, it was checked that all the various leds on the board was working in order to verify that the board actually was powered right, and that there was
no short circuts on the power grid itself.

Some of the earliest test were also to check that the FPGA actually was working properly, and it was done by making a simple FPGA echo program to test the various pins on the fpga.
The pins on the fpga were tested by connecting a led to the various FPGA-headers. If the fpga worked correctly, the led will activate, indicating the the pins actually are operating right.
When this test was conducted on the first board that were soldered, it came out that the FPGA was not "baked on" right, and that we had to start solder a new board. 

--picture of fpga headers
\todo{the thing where we made a simple fpga echo program for the pins, and tested the lines from the fpga to the headers using an led. for results: we discovered a bad bake this way}

\section{Testing IO}
\subsection{IO device tests}
\test
{Buttons \& LEDs}{
    \item{Upload a program reading the state of all buttons and turning off the LEDs corresponding to the buttons pressed down while leaving the rest of the LEDs on}
    \item{Try pressing the different buttons}
}{All LEDs light up initially and turn off when the corresponding button is pressed.}
{All LEDs light up initially and turn off when the corresponding button is pressed.}

\test
{Debug connection test}{
    \item{Connect the debug pins to the appropriate pins on the energy micro development kit}
    \item{Turn the debug to OUT}
    \item{Connect to the development kit using energyAware commander}
}{EFM32GG990F1024 listed as microcontroller}
{EFM32GG990F1024 listed as microcontroller}

\test
{SD Card test}{
    \item{Edit the code from AN0030~\cite{an0030} to use correct pins}
    \item{Compile the code, upload and run it, with SD Card connected}
    \item{Confirm data on SD Card}
}{File with string ``EFM32 ...the world's most energy friendly microcontrollers !'' is added to the SD Card.}
{The SD card was not found, and the text not present on the SD Card}

\test
{USB test}{
    \item{Compile the code from AN0065~\cite{an0065}}
    \item{Upload and run it}
    \item{Run the supplied host PC program while connected to the PCB through USB}
}{The host programs runs successfully}
{The host program fails to connect through USB}

\test
{Serial test}{
    \item{Compile the code from AN0045~\cite{an0045}}
    \item{Upload and run it}
    \item{Connect the host PC to the PCB and run terminal emulator of choice}
}{"Energy Micro RS-232 - Please press a key" appears in the terminal}
{No output in terminal}

\subsection{FPGA bus}
\test
{SRAM test}{
    \item{Write a value to a range of addresses}
    \item{Read the same address and compare with the value written}
}{The values are identical}
{Most of the values are identical, with some addresses reporting the wrong value}

\test
{Running a program}{
    \item{Upload a program to fill the memory with fibonacci numbers}
    \item{Let the CPU run for a while to ensure that somehting has been written to memory.}
    \item{Read memory and check whether the fibonacci numbers are stored, the first on adress 0, the next on the next address and so on.}
}{A sequence of fibonacci numbers in the memory.}
{Seemingly random data read from the memory}


\section{Testing Additional Components}

\subsection{Galapagos Assembler}

\todo{galapagos-as has a test-suite, write about it}

\section{Additional Tests}

\subsection{The Pseudo-Random Number Generator}

A key component in any genetic algorithm worth its salt is a decent source of (pseudo-)random numbers.
The Barricelli computer has a hardware pseudo-random number generator module built into its genetics accelerator.
When designing a pseudo-random number generator, there is always a trade-off between generating ``good'' random numbers, and generating them fast.
Having high performance as a design goal\cn, it was desirable to design a pseudo-random number generator that is as fast as possible while still meeting the minimum requirements for randomness that is needed for successfully using it in a genetics algorithm application.

The pseudo-random number generator designed for the Barricelli has been tested extensively with a pseudo-random number generator test suite called DieHarder\cn.
DieHarder is a test suite which measures the ``goodness'' of a pseudo-random number generator based a number of criteria.
\todo{ what are these criteria?}

The algorithm was implemented in python and tested agains the DieHarder integration suite\cn.

The shift-based algorithm used in the pseudo-random number generator scores quite poorly in the DieHarder tests when every single bit of the output is used.
However, by only using every 7th number\cn, the algorithm ranks quite well.

Finally, some genetics algorithms convergence tests were run, also simulated in python, using the different pseudo-random algorithm candidates as a random number source in the experiments.
Based on the results from these experiments, it is safe to conclude that, while Barricelli's pseudo-random number generator algorithm may not be best-in-class for producing convincing randomness, it is definitely good enough for problem solving using genetic algorithms, and most certainly quicker than other more ``proper'' algorithms.

\todo{ dig up some numbers, show some graphs}



\todo{The contiunous ga test}
