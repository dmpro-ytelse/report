\section{Testing the FPGA}

\subsection{Testing the fitness core}
In order to test the fitness core independent from the other parts of the architecture a simple test framework was developed. \todo{this}

\todo{we're talking about test\_utils here, right? we can cook https://github.com/dmkons/reports}

\subsubsection{Conformance tests}

\todo{the title of the test should be above the table in which its results are displayed, not just as the caption (rendered below) of the table}

\begin{table}[H]
  \begin{tabular}{r | p{8cm}}
    \noalign{\smallskip}\hline\noalign{\smallskip}
    
    What to test:  & An accurate description of which part of the system is being
    tested. \\

    \noalign{\smallskip}\hline\noalign{\smallskip}

    How to test:   & A step by step walk-through of how the conformance test is
    carried out. \\

    \noalign{\smallskip}\hline\noalign{\smallskip}

    Pass criteria: & A list of criteria which must be met in order to consider
    the conformance test as successfully passed. \\

    \noalign{\smallskip}\hline\noalign{\smallskip}
    
    Results: &  What are the result of the test. \\
   \noalign{\smallskip}\hline\noalign{\smallskip}
  
  
  
  \end{tabular}
  \caption{Conformance Test Template}
  \label{testing:fitness:pipeline_test}
\end{table}



\begin{table}[H]
  \begin{tabular}{r | p{8cm}}
    \noalign{\smallskip}\hline\noalign{\smallskip}
    
    What to test:  & Observe that RRI and RRR instructions propagate correctly through the pipeline, 
                     and produce the correct result.\\

    \noalign{\smallskip}\hline\noalign{\smallskip}

    How to test:  & The program in listing \ref{testing:listing:rrr-rri}, consisting of both RRR and RRI instructions,
                    are loaded into memory with the test framework. The execution of the instructions are observed with
                    ISim.\\

    \noalign{\smallskip}\hline\noalign{\smallskip}

    Pass criteria: &  Register r1 should contain $0xBA1212ICECC1$
    and register r2 should contain $0xBA1212ICECC1$ \\
    
     \noalign{\smallskip}\hline\noalign{\smallskip}

    Results: & The contents of register r1 and r2 are according to the pass criteria. \\
   \noalign{\smallskip}\hline\noalign{\smallskip}
  
  
  \end{tabular}
  \caption{RRR and RRI instructions}
  \label{testing:fitness:pipeline_test}
\end{table}



\begin{table}[H]
  \begin{tabular}{r | p{8cm}}
    \noalign{\smallskip}\hline\noalign{\smallskip}
    
    What to test:  & Check if the branch address is calculated correctly, and an conditional
                     jump is performed to this address. \\

    \noalign{\smallskip}\hline\noalign{\smallskip}

    How to test:   &  The program in listing \todo{create listing}, is loaded into a test bench.
                      This simple program consists of a simple loop performing some arithmetic
                      operations that store values to registers. The execution of the
                      program is simulated with isim to verify the result \\

    \noalign{\smallskip}\hline\noalign{\smallskip}

    Pass criteria: &   \\

    \noalign{\smallskip}\hline\noalign{\smallskip}
    
    Results: &  \\
   \noalign{\smallskip}\hline\noalign{\smallskip}
  
  
  
  \end{tabular}
  \caption{Branch taken}
  \label{testing:fitness:branch_taken}
\end{table}


\begin{table}[H]
  \begin{tabular}{r | p{8cm}}
    \noalign{\smallskip}\hline\noalign{\smallskip}
    
    What to test:  & Check if the conditional jump is disregarded when performing conditional
                     that always evaluate to false.\\

    \noalign{\smallskip}\hline\noalign{\smallskip}

    How to test:   &  The program in listing \todo{create listing}, is loaded into a test bench. 
                       The simple program consists of conditionals that evaluate to false. The
                       execution of the program is simulated with ISim, and the results are
                       verified. \\

    \noalign{\smallskip}\hline\noalign{\smallskip}

    Pass criteria: & Execution of the program does not store any data to the registers.
    The instructions are flushed.\\

    \noalign{\smallskip}\hline\noalign{\smallskip}
    
    Results: &   No data is stored to registers. Instruction are invalidated.\\
   \noalign{\smallskip}\hline\noalign{\smallskip}
  
  \end{tabular}
  \caption{Branch not taken}
  \label{testing:fitness:branch_not_taken}
\end{table}


\begin{table}[H]
  \begin{tabular}{r | p{8cm}}
    \noalign{\smallskip}\hline\noalign{\smallskip}
    
    What to test:  &  \\

    \noalign{\smallskip}\hline\noalign{\smallskip}

    How to test:   & \\

    \noalign{\smallskip}\hline\noalign{\smallskip}

    Pass criteria: & \\

    \noalign{\smallskip}\hline\noalign{\smallskip}
    
    Results: &  \\
   \noalign{\smallskip}\hline\noalign{\smallskip}
  
  
  
  \end{tabular}
  \caption{Conditional instruction executed }
  \label{testing:fitness:pipeline_test}
\end{table}

\begin{table}[H]
  \begin{tabular}{r | p{8cm}}
    \noalign{\smallskip}\hline\noalign{\smallskip}
    
    What to test:  & Check if conditional instructions are executed when they always evaluate to 
                     false. \\

    \noalign{\smallskip}\hline\noalign{\smallskip}

    How to test:   &  The program in listing \ref{testing:listing:conditional-not-executed}, is loaded into a test bench. 
                       The simple program consists of a set of simple conditional instructions that         
                       always evaluate to false. The execution of the program is observed in 
                       ISim,and the results are verified. The content of register r1 is observed.\\

    \noalign{\smallskip}\hline\noalign{\smallskip}

    Pass criteria: & The second instruction, the conditional \emph{ADDI}, is not executed. The content of register r1 is 1.  \\

    \noalign{\smallskip}\hline\noalign{\smallskip}
    
    Results: & The content of register r1 is 1. The conditional \emph{ADDI} instruction is not executed. .  \\
   \noalign{\smallskip}\hline\noalign{\smallskip}
  
  
  
  \end{tabular}
  \caption{Conditional instruction not executed}
  \label{testing:fitness:conditional_not_taken}
\end{table}


\begin{table}[H]
  \begin{tabular}{r | p{8cm}}
    \noalign{\smallskip}\hline\noalign{\smallskip}
    
    What to test:  & Observe that LOAD instructions is able to read from memory, and load the
                     memory content into the specified registers.  \\

    \noalign{\smallskip}\hline\noalign{\smallskip}

    How to test:   & The program in listing \todo{create listing}, consisting  of $LOAD$ and $STORE$
                     instructions. These are loaded into a test bench, and simulated with 
                     ISim. \\
                     

    \noalign{\smallskip}\hline\noalign{\smallskip}

    Pass criteria: &  The stored values is loaded from memory and stored in registers. 
                      The values in the register corresponds to the data written and loaded 
                      from memory. \\

    \noalign{\smallskip}\hline\noalign{\smallskip}
    
    Results: &  Success!\\
   \noalign{\smallskip}\hline\noalign{\smallskip}
  
  
  
  \end{tabular}
  \caption{Load data}
  \label{testing:fitness:load_data}
\end{table}


\begin{table}[H]
  \begin{tabular}{r | p{8cm}}
    \noalign{\smallskip}\hline\noalign{\smallskip}
    
    What to test:  & \\

    \noalign{\smallskip}\hline\noalign{\smallskip}

    How to test:   & \\

    \noalign{\smallskip}\hline\noalign{\smallskip}

    Pass criteria: & \\

    \noalign{\smallskip}\hline\noalign{\smallskip}
    
    Results: &  \\
   \noalign{\smallskip}\hline\noalign{\smallskip}
  
  
  
  \end{tabular}
  \caption{Store data}
  \label{testing:fitness:store_data}
\end{table}


\begin{table}[H]
  \begin{tabular}{r | p{8cm}}
    \noalign{\smallskip}\hline\noalign{\smallskip}
    
    What to test:  & Observe that STORE GENE instructions are able to store gene and fitness values
                     to the rated pool.  \\

    \noalign{\smallskip}\hline\noalign{\smallskip}

    How to test:   &  The program in listing \ref{testing:listing:store-gene}, consisting mainly of 
                       STORE GENE instructions. These are loaded into a test bench, and 
                       simulated with ISim. The content of the rated pool is verified. This is verified by performing a memory dump of the rated pool.
                      \\

    \noalign{\smallskip}\hline\noalign{\smallskip}

    Pass criteria: &  The memory dump contains an individual corresponding to the value 1 and with the corresponding fitness value of 2. These values are easily spotted when the initial values of the pool are randomly generated.     \\

    \noalign{\smallskip}\hline\noalign{\smallskip}
    
    Results: & The store of the fitness and gene is confirmed by the memory dump.   \\
   \noalign{\smallskip}\hline\noalign{\smallskip}
  
  
  
  \end{tabular}
  \caption{Store gene}
  \label{testing:fitness:store_gene}
\end{table}


\begin{table}[H]
  \begin{tabular}{r | p{8cm}}
    \noalign{\smallskip}\hline\noalign{\smallskip}
    
    What to test:  &  Observe that a gene is fetched from the unrated code, and stored in the
                      specified register\\

    \noalign{\smallskip}\hline\noalign{\smallskip}

    How to test:   &  The program in listing \todo{create listing}, consisting of LOAD GENE
                      instructions. These are loaded into a test bench and simulated with 
                      isim.
                      The content of the  un-rated pool is checked against the data loaded to
                      the fitness core. \\

    \noalign{\smallskip}\hline\noalign{\smallskip}

    Pass criteria: &  The data fetched from the rated pool is the same gene transmitted to the
                      fitness core. \\

    \noalign{\smallskip}\hline\noalign{\smallskip}
    
    Results: &  Success \\
   \noalign{\smallskip}\hline\noalign{\smallskip}
  
  
  
  \end{tabular}
  \caption{Load gene}
  \label{testing:fitness:load_gene}
\end{table}


\begin{table}[H]
\center
\begin{tabular}{|l | c | c | c | c |c|}
    \hline
    Clock cycle & IF & ID & EX & MEM & WB \\
    \hline
    3 & ADD 3, 1, 2  & ADDI 2, 10 & ADDI 1, 10   &              &              \\
    4 &              & ADD 3, 1, 2  & ADDI 2, 10    & ADDI 1, 10   &              \\
    5 &              &              & ADD 3, 1, 2   & ADDI 2, 10   & ADDI 1, 10    \\
    6 &              &              &               & ADD 3, 1, 2  & ADDI 2, 10    \\
    7 &              &              &               &              & ADD 3, 1, 2  \\
    \hline
\end{tabular}
\caption{Instruction flow}
\label{testing:tbl:instrflow}
\end{table}





\subsection{Unit Tests}

\subsection{The Pseudo-Random Number Generator}

A key component in any genetic algorithm worth its salt is a decent source of (pseudo-)random numbers.
The Barricelli computer has a hardware pseudo-random number generator module built into its genetics accelerator.
When designing a pseudo-random number generator, there is always a trade-off between generating ``good'' random numbers, and generating them fast.
Having high performance as a design goal\cn, it was desirable to design a pseudo-random number generator that is as fast as possible while still meeting the minimum requirements for randomness that is needed for successfully using it in a genetics algorithm application.

The pseudo-random number generator designed for the Barricelli has been tested extensively with a pseudo-random number generator test suite called DieHarder\cn.
DieHarder is a test suite which measures the ``goodness'' of a pseudo-random number generator based a number of criteria.
\todo{ what are these criteria?}

The algorithm was implemented in python and tested agains the DieHarder integration suite\cn.

The shift-based algorithm used in the pseudo-random number generator scores quite poorly in the DieHarder tests when every single bit of the output is used.
However, by only using every 7th number\cn, the algorithm ranks quite well.

Finally, some genetics algorithms convergence tests were run, also simulated in python, using the different pseudo-random algorithm candidates as a random number source in the experiments.
Based on the results from these experiments, it is safe to conclude that, while Barricelli's pseudo-random number generator algorithm may not be best-in-class for producing convincing randomness, it is definitely good enough for problem solving using genetic algorithms, and most certainly quicker than other more ``proper'' algorithms.

\todo{ dig up some numbers, show some graphs}

\subsubsection{VHDL Test Benches}

Unit testing VHDL entites is extremely important in a large and complex design like the Barricelli.
For this project, almost every component, perhaps except the most trivial entities, is tested in an automated or semi-automated VHDL test bench.
A tool was developed to ease the automation of VHDL test running and validation, modeled after the leading test runners in the software industry, such as JUnit\cn and Karma\cn.
This tool enabled tests to be written using easy-to-use self-evaluating tests that compare signals at specific times against expected values.

The toplevel simulation test bench of the barricelli computer, which simulates the entire FPGA as a black box interfacing against the external components, supports pre-loading entire programs into a mocked instruction memory component.
The \Gls{galapagos assembler} supports outputting assembled programs compiled to one of these mock memory components, meaning that testing new programs in a simulated environment is an easy and fun process.


\test
{Control unit test bench}{
    \item{Run the control unit test bench simulation in ISim.}
    \item{Run the control unit test bench simulation in ISim.}
    \item{Run the control unit test bench simulation in ISim.}
}{Simulation raises no assertion errors.}
{Simulation raises no assertion errors.}

\test
{Control unit test bench}
{
\item{Hello world.}
}
{Simulation raises no assertion errors.}
{Simulation raises no assertion errors.}

\subsection{Integration Tests}
The system is module based with the different components of placed in logically related groups. The most important groups are the \emph{genetic pipeline}, the \emph{fitness core}, the \emph{Instruction controller} and the \emph{Data controller}. Combined these component groups represent the entire system. These components are of course subject of 

\subsection{Validation Tests}

\section{Testing the PCB}
During and after the components were soldered on the PCB board, the board were tested to ensure that the powergrid were working as it was supposed to.
For the first test, it was checked that all the various leds on the board was working in order to verify that the board actually was powered right, and that there was
no short circuts on the power grid itself.

Some of the earliest test were also to check that the FPGA actually was working properly, and it was done by making a simple FPGA echo program to test the various pins on the fpga.
The pins on the fpga were tested by connecting a led to the various FPGA-headers. If the fpga worked correctly, the led will activate, indicating the the pins actually are operating right.
When this test was conducted on the first board that were soldered, it came out that the FPGA was not "baked on" right, and that we had to start solder a new board. 

--picture of fpga headers
\todo{the thing where we made a simple fpga echo program for the pins, and tested the lines from the fpga to the headers using an led. for results: we discovered a bad bake this way}

\section{Testing IO}

\section{Testing Additional Components}

\subsection{Galapagos Assembler}

\todo{galapagos-as has a test-suite, write about it}

\section{Additional Tests}


\todo{The contiunous ga test}
