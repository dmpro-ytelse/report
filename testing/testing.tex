When designing a computer with a custom architecture from scratch, it is important to continually test and evaluate the correctness of the solution at all possible stages, to ensure that final product is a success.
This section documents and explains the rationale behind the different types of tests that have been performed.

\section{Testing the Processor}

Barricelli's processor has been tested at four different levels: \gls{VHDL}-based unit test simulations of the different subcomponents,  \gls{VHDL}-based integration test simulations of each processing unit, \gls{VHDL}-based system test simulations of the entire system interfacing against a mock SCU and mock memory, and finally physical integration tests of the processor programmed onto the FPGA of the Barricelli.

\todo{what about timing simulations?}

\subsection{\gls{VHDL}-based Subcomponent Unit Test Simulations}

Unit testing VHDL entities is extremely important in a large and complex design like the Barricelli.
For this project, almost every component, perhaps except the most trivial entities, is tested in an automated or semi-automated VHDL test bench.
A tool was developed to ease the automation of VHDL test running and validation, modeled after the leading test runners in the software industry, such as JUnit\cite{junit} and Karma\cite{karma}.
This tool enabled tests to be written using easy-to-use self-evaluating tests that compare signals at specific times against expected values.

The goal of these unit tests is to ensure that the building block components work as expected when reacting to specified input.

Screenshots of simulations of these tests can be found in Appendix \ref{appendix:test-bench-documentation}.
\todo{ Make sure screenshots of as many tests as possible are available. If not, then we probably have to do references from the following tables}
\todo{we need to stash results somewhere}.

\subsubsection{Fitness Core Components}
\todo{ Insert text and tables for components used in Fitness Core}

\subsubsection{Genetic Pipeline Components}
\todo{ Insert text and tables for components used in Genetics Pipeline}
\paragraph{Selection Core}
\paragraph{Genetic Pipeline Controller?}
\paragraph{Unrated pool?}
\paragraph{Rated pool?}

\begin{table}[H]
  \begin{tabular}{r | p{9cm}}
    \noalign{\smallskip}\hline\noalign{\smallskip}
    
    What to test:  & Check if crossover split function performs crossover correctly, 
                        from correct bit \\

    \noalign{\smallskip}\hline\noalign{\smallskip}

    How to test:   &    Changes in any input should cause change in the outputs.
                        Therefore parent inputs and random\_number will be changed
                        during test. 
                        \\
                      
    \noalign{\smallskip}\hline\noalign{\smallskip}

    Pass criteria: &    The output for child1 should have output from parent1 and child2
                        from parent2 before crossover point, and child1 should have
                        output from parent2 and child2 from parent1 after crossover
                        point. 
                        The starting point, which is the first bit in the crossover,
                        should always be the bit number equal to the value of
                        random\_number.
                        \\
    \noalign{\smallskip}\hline\noalign{\smallskip}
    
    Results: &      Successful. 
                    Changes in parents cause expected changes in children, and starting 
                    point for crossover is always equal to the value of random\_number
                    \\
   \noalign{\smallskip}\hline\noalign{\smallskip}
  
  
  
  \end{tabular}
  \caption{Crossover Core Split function}
  \label{testing:components:genetic_pipeline:crossover_core_split}
\end{table}

\begin{table}[H]
  \begin{tabular}{r | p{9cm}}
    \noalign{\smallskip}\hline\noalign{\smallskip}
    
    What to test:  & Check if Crossover Double-Split Function performs crossover
                     correctly, from correct starting bit to correct ending bit \\

    \noalign{\smallskip}\hline\noalign{\smallskip}

    How to test:   &    Changes in any input should cause change in the outputs.
                        Therefore parent inputs and random\_numbers will be changed
                        during test.  
                        \\
                      
    \noalign{\smallskip}\hline\noalign{\smallskip}

    Pass criteria: &    The output for child1 should have output from parent1 and child2
                        from parent2 before crossover starting point and after ending 
                        point, and child1 should have output from parent2 and child2
                        from parent1 between the crossover starting point and ending
                        point. 
                        The random\_number with the highest value should always be the 
                        starting point, and the one with the lowest value should always
                        be the ending point. 
                        These points, which are the first and the last bit in the 
                        crossover, should always be the bit numbers equal to the value     
                        of the random\_numbers. 
                        If both have same value, then only one bit location will have a
                        crossover
                        \\
    \noalign{\smallskip}\hline\noalign{\smallskip}
    
    Results: &      Successful. 
                    Changes in parents cause expected changes in children, and starting 
                    point for crossover is always equal to the value of the highest 
                    random\_number, and ending point for crossover is always equal to 
                    the value of the lowest random\_number
                    \\
   \noalign{\smallskip}\hline\noalign{\smallskip}
  
  
  
  \end{tabular}
  \caption{Crossover Core Double-Split function}
  \label{testing:components:genetic_pipeline:crossover_core_doublesplit}
\end{table}

\begin{table}[H]
  \begin{tabular}{r | p{9cm}}
    \noalign{\smallskip}\hline\noalign{\smallskip}
    
    What to test:  & Check if Crossover XOR Function performs crossover
                     correctly, from correct starting bit to correct ending bit \\

    \noalign{\smallskip}\hline\noalign{\smallskip}

    How to test:   &    Changes in any input should cause change in the outputs.
                        Therefore parent inputs and random\_number will be changed
                        during test.  
                        \\
                      
    \noalign{\smallskip}\hline\noalign{\smallskip}

    Pass criteria: &    The output for child1 should have output from parent1 and child2
                        from parent2 for each bit \emph{i}, where in the random\_number 
                        the value is 0, and child1 should have output from parent2 and 
                        child2 from parent1 for each bit \emph{i}, where in the 
                        random\_number the value is 1.
                        \\
    \noalign{\smallskip}\hline\noalign{\smallskip}
    
    Results: &      Successfull.
                    Changes in parents cause expected changes in children, and for each
                    bit \emph{i} in the random\_number, there are crossover at same bit    
                    \emph{i} from the parents to the children.
                    \\
   \noalign{\smallskip}\hline\noalign{\smallskip}
  
  
  
  \end{tabular}
  \caption{Crossover Core XOR function}
  \label{testing:components:genetic_pipeline:crossover_core_xor}
\end{table}

\begin{table}[H]
  \begin{tabular}{r | p{8cm}}
    \noalign{\smallskip}\hline\noalign{\smallskip}
    
    What to test:  & Check if Crossover Toplevel selects correct crossover function
                     based on control\_input, and random\_number when in "Party Mode"\\

    \noalign{\smallskip}\hline\noalign{\smallskip}

    How to test:   &    Changes in parents are not relevant, since this test is not 
                        inteded to test the functions themselves, only the function 
                        selection.
                        Every input of control\_input will be tested.
                        Changes in random\_number will be done with focus on the 2 LS 
                        bits when control\_input is "1XX", and in party mode.
                        \\
                      
    \noalign{\smallskip}\hline\noalign{\smallskip}

    Pass criteria: &    When control\_input is set to 000, or 1XX and random\_input-bits 
                        to 00, crossover should be split with the value of the 6 LS bits 
                        from random\_number used for starting point.
                        When control\_input is set to 001, or 1XX and random\_input-bits 
                        to 01, crossover should be double-split, with the value of the 
                        12 LS bits from random\_number used for starting and ending 
                        point.
                        When control\_input is set to 010, or 1XX and random\_input-bits 
                        to 10, crossover should be xor, with crossover on every bit 
                        numbers that are 1 in random\_number.
                        When control\_input is set to 011, or 1XX and random\_input-bits 
                        to 11 there should be no crossover at all, and output children 
                        should be equal to each their input parent.
                        \\
    \noalign{\smallskip}\hline\noalign{\smallskip}
    
    Results: &      Successfull. 
                    Each value in control\_input was tested, and set the expected 
                    function. When set to 1XX, every value on the 2 LS bits in the 
                    random\_number was tested, and set the expected function.
                    \\
   \noalign{\smallskip}\hline\noalign{\smallskip}
  
  
  
  \end{tabular}
  \caption{Crossover Core Toplevel}
  \label{testing:components:genetic_pipeline:crossover_core_toplevel}
\end{table}

\begin{table}[H]
  \begin{tabular}{r | p{9cm}}
    \noalign{\smallskip}\hline\noalign{\smallskip}
    
    What to test:  & Check if Mutation Core selects mutates when allowed, mutates the 
                     correct amount of bits, and the correct bit numbers, all based
                     on chance\_input and random\_number.\\

    \noalign{\smallskip}\hline\noalign{\smallskip}

    How to test:   &    Changes on input will change output. Therefore input will have 
                        changes.
                        Changes in random\_number and chance\_input  will be done with
                        focus to test allowing or denying mutation.
                        Changes in random\_number will also be done to test amount of 
                        allowed mutations, and to test selecting the locations of the 
                        mutations
                        \\
                      
    \noalign{\smallskip}\hline\noalign{\smallskip}

    Pass criteria: &    When the P first bits in random\_number is equal to or higher 
                        than chance\_input (size P), there should be no mutation at all.
                        When mutation is allowed, the next two bits should allow these 
                        amount of mutations: 1-4 depending on values 00-11.
                        Bits 23-0 select four bit locations for mutations, and the 
                        output should have opposite value on these locations compared to
                        the input.
                        If more than one bit location pointer has the same value, the 
                        same bit location should still have the mutation on the output.
                                                \\
    \noalign{\smallskip}\hline\noalign{\smallskip}
    
    Results: &      Successful. 
                    Mutation is allowed only when the P first bits are lower than the
                    chance\_input, the correct amount of mutations were set and each 
                    four bit locations were selected correctly as expected by bits 23-0
                    \\
   \noalign{\smallskip}\hline\noalign{\smallskip}
  
  
  \end{tabular}
  \caption{Mutation Core}
  \label{testing:components:genetic_pipeline:mutation_core}
\end{table}



\subsubsection{Other Components}
\todo{ Insert text and tables for components used elsewhere}

\subsection{\gls{VHDL}-based Processing Unit Integration Test Simulations}

Each processing unit, which each consists of several interconnected subcomponents, has been simulated for integration testing.
The goal of these tests are to verify that the different subcomponents interface correctly with each other, and that the behaviour of the supercomponent is as expected.

\subsubsection{Testing the Fitness Core}

\todo{about tb\_fitness\_core.vhd}

\subsubsection{Testing the Genetics Pipeline}

\todo{about tb\_genetics\_pipeline.vhd}

\subsection{\gls{VHDL}-based System test Simulations}
\label{section:testing:fpga:system-tests}

The toplevel simulation test bench of the Barricelli computer, which simulates the entire FPGA as a black box interfacing against the external components, supports pre-loading entire programs into a mocked instruction memory component.
The \Gls{galapagos assembler} supports outputting assembled programs compiled to one of these mock memory components, meaning that testing new programs in a simulated environment is an easy and fun process.

\todo{the title of the test should be above the table in which its results are displayed, not just as the caption (rendered below) of the table}

A formal description of the system tests performed at this level can be found in tables
\ref{testing:fitness:pipeline_test},
\ref{testing:fitness:branch_taken},
\ref{testing:fitness:branch_not_taken},
\ref{testing:fitness:conditional_taken},
\ref{testing:fitness:conditional_not_taken},
\ref{testing:fitness:load_data},
\ref{testing:fitness:store_data},
\ref{testing:fitness:store_gene},
and
\ref{testing:fitness:load_gene}.

\begin{table}[H]
  \begin{tabular}{r | p{8cm}}
    \noalign{\smallskip}\hline\noalign{\smallskip}
    
    What to test:  & Observe that RRI and RRR instructions propagate correctly through the pipeline, 
                     and produce the correct result.\\

    \noalign{\smallskip}\hline\noalign{\smallskip}

    How to test:  & The program in listing \todo{create listing}, consisting of both RRR and RRI instructions,
                    are loaded into memory with the test framework. The execution of the instructions are observed with
                    isim.\\

    \noalign{\smallskip}\hline\noalign{\smallskip}

    Pass criteria: & The flow of data is according to the architecture presented in figure. \todo{add reference}
                   Register 1, 2, 3 are loaded with 9, 10 and 19, respectively.   \\
    
     \noalign{\smallskip}\hline\noalign{\smallskip}

    Results: &  What are the result of the test. \\
   \noalign{\smallskip}\hline\noalign{\smallskip}
  
  
  \end{tabular}
  \caption{RRR and RRI instructions}
  \label{testing:fitness:pipeline_test}
\end{table}


\begin{table}[H]
  \begin{tabular}{r | p{8cm}}
    \noalign{\smallskip}\hline\noalign{\smallskip}
    
    What to test:  & Check if the branch address is calculated correctly, and an conditional
                     jump is performed to this address. \\

    \noalign{\smallskip}\hline\noalign{\smallskip}

    How to test:   &  The program in listing \todo{create listing}, is loaded into a test bench.
                      This simple program consists of a simple loop performing some arithmetic
                      operations that store values to registers. The execution of the
                      program is simulated with isim to verify the result \\

    \noalign{\smallskip}\hline\noalign{\smallskip}

    Pass criteria: &  The branch is taken. The instructions located in the $fetch stage$, 
                       $decode stage$, and $execute stage$ are flushed. The results in registers
                       should be X, X, and X in registers X, X and X, respectively. \\

    \noalign{\smallskip}\hline\noalign{\smallskip}
    
    Results: &  Registers X, X, and X is contained in the registers. The instructions in $fetch
                stage$, $decode stage$ and $execute stage$ does not perform any  
                changes to the register file. \\
   \noalign{\smallskip}\hline\noalign{\smallskip}
  
  
  
  \end{tabular}
  \caption{Branch taken}
  \label{testing:fitness:branch_taken}
\end{table}

\begin{table}[H]
  \begin{tabular}{r | p{8cm}}
    \noalign{\smallskip}\hline\noalign{\smallskip}
    
    What to test:  & Check if the conditional jump is disregarded when performing conditional
                     that always evaluate to false.\\

    \noalign{\smallskip}\hline\noalign{\smallskip}

    How to test:   &  The program in listing \todo{create listing}, is loaded into a test bench. 
                       The simple program consists of conditionals that evaluate to false. The
                       execution of the program is simulated with isim, and the results are
                       verified. \\

    \noalign{\smallskip}\hline\noalign{\smallskip}

    Pass criteria: & Execution of the program should not store any data to the registers.\\

    \noalign{\smallskip}\hline\noalign{\smallskip}
    
    Results: &   No data is stored to registers. \\
   \noalign{\smallskip}\hline\noalign{\smallskip}
  
  
  
  \end{tabular}
  \caption{Branch not taken}
  \label{testing:fitness:branch_not_taken}
\end{table}

\begin{table}[H]
  \begin{tabular}{r | p{8cm}}
    \noalign{\smallskip}\hline\noalign{\smallskip}
    
    What to test:  & Check if conditional instruction are executed when they
                     always are evaluated to true.   \\

    \noalign{\smallskip}\hline\noalign{\smallskip}

    How to test:  & The program in listing \todo{create listing}, is loaded into test bench. The 
                    simple program consists of a set with simple conditional instructions that
                    always evaluate to true. The execution of the program is simulated with isim, 
                    and the result is verified. 
    \\

    \noalign{\smallskip}\hline\noalign{\smallskip}

    Pass criteria: & The instructions propagates normally through the pipeline. The different instruction are executed and their results are written to the register file. \\

    \noalign{\smallskip}\hline\noalign{\smallskip}
    
    Results: &  \\
   \noalign{\smallskip}\hline\noalign{\smallskip}
  
  
  
  \end{tabular}
  \caption{Conditional instruction executed }
  \label{testing:fitness:conditional_taken}
\end{table}

\begin{table}[H]
  \begin{tabular}{r | p{8cm}}
    \noalign{\smallskip}\hline\noalign{\smallskip}
    
    What to test:  & Check if conditional instructions are executed when they always evaluate to 
                     false. \\

    \noalign{\smallskip}\hline\noalign{\smallskip}

    How to test:   &  The program in listing \ref{testing:listing:conditional-not_executed}, is loaded into a test bench. 
                       The simple program consists of a set of simple conditional instructions that         
                       always evaluate to false. The execution of the program is observed in 
                       ISim,and the results are verified. The content of register r1 is observed.\\

    \noalign{\smallskip}\hline\noalign{\smallskip}

    Pass criteria: & The second instruction, the conditional \emph{ADDI}, is not executed. The content of register r1 is 1.  \\

    \noalign{\smallskip}\hline\noalign{\smallskip}
    
    Results: & The content of register r1 is 1. The conditional \emph{ADDI} instruction is not executed. .  \\
   \noalign{\smallskip}\hline\noalign{\smallskip}
  
  
  
  \end{tabular}
  \caption{Conditional instruction not executed}
  \label{testing:fitness:conditional_not_taken}
\end{table}

\begin{table}[H]
  \begin{tabular}{r | p{8cm}}
    \noalign{\smallskip}\hline\noalign{\smallskip}
    
    What to test:  & \\

    \noalign{\smallskip}\hline\noalign{\smallskip}

    How to test:   & \\

    \noalign{\smallskip}\hline\noalign{\smallskip}

    Pass criteria: & \\

    \noalign{\smallskip}\hline\noalign{\smallskip}
    
    Results: &  \\
   \noalign{\smallskip}\hline\noalign{\smallskip}
  
  
  
  \end{tabular}
  \caption{Store data}
  \label{testing:fitness:pipeline_test}
\end{table}
\begin{table}[H]
  \begin{tabular}{r | p{8cm}}
    \noalign{\smallskip}\hline\noalign{\smallskip}
    
    What to test:  & Observe that LOAD instructions is able to read from memory, and load the
                     memory content into the specified registers.  \\

    \noalign{\smallskip}\hline\noalign{\smallskip}

    How to test:   & The program in listing \todo{create listing}, consisting mainly of LOAD
                     instructions. These are loaded into a testbench, and simulated with 
                     isim. \\
                     

    \noalign{\smallskip}\hline\noalign{\smallskip}

    Pass criteria: &  \\

    \noalign{\smallskip}\hline\noalign{\smallskip}
    
    Results: &  \\
   \noalign{\smallskip}\hline\noalign{\smallskip}
  
  
  
  \end{tabular}
  \caption{Load data}
  \label{testing:fitness:load_data}
\end{table}

\begin{table}[H]
  \begin{tabular}{r | p{8cm}}
    \noalign{\smallskip}\hline\noalign{\smallskip}
    
    What to test:  &  \\

    \noalign{\smallskip}\hline\noalign{\smallskip}

    How to test:   &  \\

    \noalign{\smallskip}\hline\noalign{\smallskip}

    Pass criteria: &  \\

    \noalign{\smallskip}\hline\noalign{\smallskip}
    
    Results: &  \\
   \noalign{\smallskip}\hline\noalign{\smallskip}
  
  
  
  \end{tabular}
  \caption{Store gene}
  \label{testing:fitness:store_gene}
\end{table}

\begin{table}[H]
  \begin{tabular}{r | p{8cm}}
    \noalign{\smallskip}\hline\noalign{\smallskip}
    
    What to test:  &  Observe that a gene is fetched from the unrated code, and stored in the
                      specified register\\

    \noalign{\smallskip}\hline\noalign{\smallskip}

    How to test:   &  The program in listing \ref{testing:listing:load-gene}, consisting of LOAD GENE
                      instructions. These are loaded into a test bench and simulated with 
                      ISim. The content of the location of the distributed counters are checked against the data 
                      loaded to the fitness cores.  \\

    \noalign{\smallskip}\hline\noalign{\smallskip}

    Pass criteria: &  The data fetched from the rated pool is the same gene transmitted to the
                      fitness core. \\

    \noalign{\smallskip}\hline\noalign{\smallskip}
    
    Results: &  Success \\
   \noalign{\smallskip}\hline\noalign{\smallskip}
  
  
  
  \end{tabular}
  \caption{Load gene}
  \label{testing:fitness:load_gene}
\end{table}

%%\begin{table}[H]
\center
\begin{tabular}{|l | c | c | c | c |c|}
    \hline
    Clock cycle & IF & ID & EX & MEM & WB \\
    \hline
    3 & ADD 3, 1, 2  & ADDI 2, 10 & ADDI 1, 10   &              &              \\
    4 &              & ADD 3, 1, 2  & ADDI 2, 10    & ADDI 1, 10   &              \\
    5 &              &              & ADD 3, 1, 2   & ADDI 2, 10   & ADDI 1, 10    \\
    6 &              &              &               & ADD 3, 1, 2  & ADDI 2, 10    \\
    7 &              &              &               &              & ADD 3, 1, 2  \\
    \hline
\end{tabular}
\caption{Instruction flow}
\label{testing:tbl:instrflow}
\end{table}




\begin{table}[H]
  \begin{tabular}{r | p{8cm}}
    \noalign{\smallskip}\hline\noalign{\smallskip}
    
    What to test:  &  Test a specific genetic problem using the Galapagos architecture. 
                      The problem in question aims to find a specific color, $magic pink$, by 
                      genetic evolution. \\

    \noalign{\smallskip}\hline\noalign{\smallskip}

    How to test:   &  The program in listing \ref{testing:listing:color-search} is loaded into a test bench. The
    programs consists of both genetic and fitness related instructions. Program is executed and
    verified with ISim. The registers containing the best chromosome and fitness values are studied
    during the run. \\

    \noalign{\smallskip}\hline\noalign{\smallskip}

    Pass criteria: &  Execution shall show an improvement of the fitness scores and the chromosomes
    as the program simulates. E.g that it converges against a solution. \\

    \noalign{\smallskip}\hline\noalign{\smallskip}
    
    Results: &   The problem converges and the color is found.\\
   \noalign{\smallskip}\hline\noalign{\smallskip}
  
  
  
  \end{tabular}
  \caption{Find color: A genetic solution}
  \label{testing:genetic:genetic_color}
\end{table}


\begin{table}[H]
  \begin{tabular}{r | p{8cm}}
    \noalign{\smallskip}\hline\noalign{\smallskip}
    
    What to test:  & Test a spesfic genetic problem using the barricelli computer. 
                     The problem in question aims to find a solution to the knapsack problem. 
                     The problems involves finding the best combination of items to put into a
                     knapsack with a weight constraint. The test start with a set of items with
                     a given score and weight. 
                      \\

    \noalign{\smallskip}\hline\noalign{\smallskip}

    How to test:  & The program in listing \todo{Add listing} is loaded into a test bench. 
                    The program consists of both genetic and fitness related instructions.
                    The program is executed and verified in isim. The registers containing the
                    best solutions are studied during the run. \\

    \noalign{\smallskip}\hline\noalign{\smallskip}

    Pass criteria: &  It is observed that the best solution converges against a better solution
                       regularly. E.g that it continuously improve for the better.  \\
    
     \noalign{\smallskip}\hline\noalign{\smallskip}

    Results: &   It is observable that it improve after a number of microseconds. It is, however, 
                 difficult to determine if this solution is good since the simulation 
                 is limited to just a few microseconds. Note that the simulations
                 create a lot of simulation related data for a small amount of simulation time.  \\
   \noalign{\smallskip}\hline\noalign{\smallskip}
  
  
  \end{tabular}
  \caption{The knapsack problem : A genetic solution}
  \label{testing:fitness:pipeline_test}
\end{table}




\subsection{Timing simulation}
When designing an processor architecture on hardware it is important to take timing into considerations. \todo{Write me}




\subsection{Physical Integration Tests}

Finally, on the physical board, the processor was tested by running the same programs as in the system tests described in section \vref{section:testing:fpga:system-tests}.
These programs were programmed into the instruction memory of the processor by the SCU.

\section{Testing the PCB}
During and after the components were soldered on the PCB board, the board were tested to ensure that the power grid were working as it was supposed to.
For the first test, it was checked that all the various LEDs on the board was working in order to verify that the board actually was powered right, and that there was
no short circuits on the power grid itself.

Some of the earliest test were also to check that the FPGA actually was working properly, and it was done by making a simple FPGA echo program to test the various pins on the FPGA.
The pins on the FPGA were tested by connecting a led to the various FPGA-headers. If the FPGA worked correctly, the led will activate, indicating the the pins actually are operating right.
When this test was conducted on the first board that were soldered, it came out that the FPGA was not "baked on" right, and that we had to start solder a new board. 

\subsection{testing the SD card}
After the completion of the soldering process for the PCB a test were also conducted in order to ensure that the 
SD card were connected right, and outputting the right signals. 
--picture of FPGA headers
\todo{the thing where we made a simple FPGA echo program for the pins, and tested the lines from the FPGA to the headers using an led. for results: we discovered a bad bake this way}

\section{Testing IO}
\subsection{IO device tests}
\test
{Buttons \& LEDs}{
    \item{Upload a program reading the state of all buttons and turning off the LEDs corresponding to the buttons pressed down while leaving the rest of the LEDs on}
    \item{Try pressing the different buttons}
}{All LEDs light up initially and turn off when the corresponding button is pressed.}
{All LEDs light up initially and turn off when the corresponding button is pressed.}

\test
{Debug connection test}{
    \item{Connect the debug pins to the appropriate pins on the energy micro development kit}
    \item{Turn the debug to OUT}
    \item{Connect to the development kit using energyAware commander}
}{EFM32GG990F1024 listed as microcontroller}
{EFM32GG990F1024 listed as microcontroller}

\test
{SD Card test}{
    \item{Edit the code from AN0030~\cite{an0030} to use correct pins}
    \item{Compile the code, upload and run it, with SD Card connected}
    \item{Confirm data on SD Card}
}{File with string ``EFM32 ...the world's most energy friendly microcontrollers !'' is added to the SD Card.}
{The SD card was not found, and the text not present on the SD Card}

\test
{USB test}{
    \item{Compile the code from AN0065~\cite{an0065}}
    \item{Upload and run it}
    \item{Run the supplied host PC program while connected to the PCB through USB}
}{The host programs runs successfully}
{The host program fails to connect through USB}

\test
{Serial test}{
    \item{Compile the code from AN0045~\cite{an0045}}
    \item{Upload and run it}
    \item{Connect the host PC to the PCB and run terminal emulator of choice}
}{"Energy Micro RS-232 - Please press a key" appears in the terminal}
{No output in terminal}

\subsection{FPGA bus}
\test
{SRAM test}{
    \item{Write a value to a range of addresses}
    \item{Read the same address and compare with the value written}
}{The values are identical}
{Most of the values are identical, with some addresses reporting the wrong value}

\test
{Running a program}{
    \item{Upload a program to fill the memory with fibonacci numbers}
    \item{Let the CPU run for a while to ensure that somehting has been written to memory.}
    \item{Read memory and check whether the fibonacci numbers are stored, the first on adress 0, the next on the next address and so on.}
}{A sequence of fibonacci numbers in the memory.}
{Seemingly random data read from the memory}


\section{Testing Additional Components}

\subsection{Galapagos Assembler}

\todo{galapagos-as has a test-suite, write about it}

\section{Additional Tests}

\subsection{The Pseudo-Random Number Generator}


The pseudo-random number generator designed for the Barricelli has been tested extensively with a pseudo-random number generator test suite called DieHarder\cn.
DieHarder is a test suite which measures the ``goodness'' of a pseudo-random number generator based a number of criteria.
\todo{ what are these criteria?}

The algorithm was implemented in python and tested against the DieHarder integration suite\cn.

The shift-based algorithm used in the pseudo-random number generator scores quite poorly in the DieHarder tests when every single bit of the output is used.
However, by only using every 7th number\cn, the algorithm ranks quite well.
A condensed DieHarder test result overview can be found in Table \vref{table:dieharder-results}.
The descriptions in the table are modified from the descriptions in the output of the DieHarder test suite.

\begin{table}[H]
    \begin{tabular}{| l | l |}
    
    \hline
    Test Name & Pass? \\
    \hline
    DieHard "Birthdays Test" & FAILED \\
    \hline
    Diehard Overlapping 5-Permutations Test & FAILED \\
    \hline
    Diehard 32x32 Binary Rank Test & FAILED \\
    \hline
    Diehard 6x8 Binary Rank Test & FAILED \\
    \hline
    Diehard Bitstream Test. & FAILED \\
    \hline
    Diehard Overlapping Pairs Sparse Occupance (OPSO) & FAILED \\
    \hline
    Diehard Overlapping Quadruples Sparce Occupancy (OQSO) Test & FAILED \\
    \hline
    Diehard DNA Test & FAILED \\
    \hline
    Diehard Count the 1s (stream) (modified) Test & FAILED \\
    \hline
    Diehard Count the 1s Test (byte) (modified) & FAILED \\
    \hline
    Diehard Parking Lot Test (modified) & FAILED \\
    \hline
    Diehard Minimum Distance (2d Circle) Test & FAILED \\
    \hline
    Diehard 3d Sphere (Minimum Distance) Test & FAILED \\
    \hline
    Diehard Squeeze Test & FAILED \\
    \hline
    Diehard Sums Test & WEAK \\
    \hline
    Diehard Runs Test & FAILED \\
    \hline
    Diehard Craps Test & FAILED \\
    \hline
    Marsaglia and Tsang GCD Test & FAILED \\
    \hline
    STS Monobit Test & WEAK \\
    \hline
    STS Runs Test & PASSED \\
    \hline
    STS Serial Test & WEAK \\
    \hline
    RGB Bit Distribution Test & FAILED/WEAK \\
    \hline
    the generalized minimum distance test & FAILED \\
    \hline
    RGB Permutations Test & PASSED \\
    \hline
    RGB Lagged Sums Test & PASSED \\
    \hline
    The Kolmogorov-Smirnov Test Test & WEAK \\
    \hline
    DAB Byte Distribution Test & PASSED \\
    \hline
    DCT (Frequency Analysis) Test & FAILED \\
    \hline
    DAB Fill Tree Test & FAILED \\
    \hline
    DAB Fill Tree 2 Test & FAILED \\
    \hline
    DAB Monobit 2 Test & FAILED \\
    \hline
  \end{tabular}
  \caption{DieHarder test results of the PRNG}
  \label{testing:prng:1-time}
\end{table}


Finally, some genetics algorithms convergence tests were run, also simulated in python, using the different pseudo-random algorithm candidates as a random number source in the experiments.
Based on the results from these experiments, it is safe to conclude that, while Barricelli's pseudo-random number generator algorithm may not be best-in-class for producing convincing randomness, it is definitely good enough for problem solving using genetic algorithms, and most certainly quicker than other more ``proper'' algorithms.

\todo{ dig up some numbers, show some graphs}



\todo{The continuous ga test}
