\begin{table}[H]
  \begin{tabular}{r | p{8cm}}
    \noalign{\smallskip}\hline\noalign{\smallskip}
    
    What to test:  & Check if conditional instruction are executed when they
                     always are evaluated to true.   \\

    \noalign{\smallskip}\hline\noalign{\smallskip}

    How to test:  & The program in listing \ref{testing:listing:conditional-executed}, is loaded into test bench.     The 
                    simple program consists an \emph{ADDI} and  an conditional \emph{ADDI} instruction that
                    always evaluate to true. The execution of the program is simulated with ISim, 
                    and the result is verified. More specifically, the content of register r1 are verified. 
    \\

    \noalign{\smallskip}\hline\noalign{\smallskip}

    Pass criteria: & The first \emph{ADDI} results in the r1 to be incremented with 1. The second
    instruction, the conditional, increments the value of r1 to 2. \\

    \noalign{\smallskip}\hline\noalign{\smallskip}
    
    Results: &  The contents of register r1 is 2. This proves that the conditional instruction was executed. \\
   \noalign{\smallskip}\hline\noalign{\smallskip}
  
  
  
  \end{tabular}
  \caption{Conditional instruction executed }
  \label{testing:fitness:conditional_taken}
\end{table}
