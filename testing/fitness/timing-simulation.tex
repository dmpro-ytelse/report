When designing an processor architecture on hardware it is important to take timing into considerations. Electric circuitry has some small delay for electric signals to propagate through the circuits. When performing normal behaviour simulation these delays are not detectable. Behaviour simulation considers circuits without delay; everything happens instant. This is fine when checking if the code behaves as intended. However, the real world is not perfect. There is need to check how this logic actually behaves on circuits with the accompanying delay. 

This is accomplished by performing what is referred to as timing simulation. During the compilation phase of the logic it is possible to generate timing data to the simulations. With this timing data the simulation is able to simulate the logic with real delay. When observing the timing simulations it is possible to actually see how the different signals propagate. By observing the simulations with these delay it is possible to uncover errors that would have been undetected during the behaviour simulation. 

In the Galapagos the logic was constructed very carefully to avoid such timing issues. This involved removing latches and being careful to synchronize the components with the clock. And not least, only use one clock to not complicate things. Because of this a very few errors were uncovered during this simulation. Actually, only one error was discovered. As it turned out there existed some delay between loading the instructions and execute them. The cycles was delayed with one cycle. This was resolved by inserting a $NOP$.  


