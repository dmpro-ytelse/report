\begin{table}[H]
  \begin{tabular}{r | p{8cm}}
    \noalign{\smallskip}\hline\noalign{\smallskip}
    
    What to test:  & Test a spesfic genetic problem using the barricelli computer. 
                     The problem in question aims to find a solution to the knapsack problem. 
                     The problems involves finding the best combination of items to put into a
                     knapsack with a weight constraint. The test start with a set of items with
                     a given score and weight. The program can be found in listing \ref{testing:listing:knapsack}.
                      \\

    \noalign{\smallskip}\hline\noalign{\smallskip}

    How to test:  & The program in listing \ref{testing:listing:knapsack} is loaded into a test bench. 
                    The program consists of both genetic and fitness related instructions.
                    The program is executed and verified in isim. The registers containing the
                    best solutions are studied during the run. \\

    \noalign{\smallskip}\hline\noalign{\smallskip}

    Pass criteria: &  It is observed that the best solution converges against a better solution
                       regularly. E.g that it continuously improve for the better.  \\
    
     \noalign{\smallskip}\hline\noalign{\smallskip}

    Results: &   It is observable that it improve after a number of microseconds. It is, however, 
                 difficult to determine if this solution is good since the simulation 
                 is limited to just a few microseconds. Note that the simulations
                 create a lot of simulation related data for a small amount of simulation time.  \\
   \noalign{\smallskip}\hline\noalign{\smallskip}
  
  
  \end{tabular}
  \caption{The knapsack problem : A genetic solution}
  \label{testing:fitness:pipeline_test}
\end{table}

