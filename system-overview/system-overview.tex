\todo{image of the finished computer}

\section{Application}

The Barricelli computer, which is pictured in Figure \todo{reference image of the finished computer} is a computer designed for solving problems using genetic algorithms.
It is highly optimized for problems for which it is possible to express individuals in 64 bits or less.
The Barricelli can be controlled through one of its external communication channels, such as USB, but can also be commandeered through the built-in buttons and LEDs.

\todo{more}

\section{System Architecture}

This section aims to provide a bird's-eye view of the system architecture of the \Gls{barricelli} computer.

\begin{figure}[H]
\includegraphics[width=\textwidth]{dmpro/Sketch 1 - System Overview.png}
\caption{System Overview}
\label{figure:system-overview}
\end{figure}
\todo{increase font size in illustration}

The requirements in section \vref{section:requirements} state that the \Gls{barricelli} should use an \gls{FPGA} and a microcontroller as system components.
The \Gls{barricelli}'s specialized CPU design is implemented and compiled onto an \gls{FPGA}, and a microcontroller is used to facilitate Input/Output, control and communications between the \gls{FPGA} and the outside world.
A graphical overview illustrating the system architecture can be found in figure \vref{figure:system-overview}.

\section{Components}

This section contains a list of the components in figure \vref{figure:system-overview}, documenting their roles in the system architecture.

\subsection{\gls{FPGA}}

The \gls{FPGA} is a Xilinx Spartan6 LX45\cn Field-Programmable Gate Array, which is a microchip that can be programmed to behave like any arbitrary intergrated circuit.
The \gls{FPGA} is programmed to contain an implementation of the custom \Gls{galapagos} Processor Architecture designed for the \Gls{barricelli} computer.
It is connected to the \gls{SCU} by a 41-bit wide bus, though which the processor can be programmed by users.
The \gls{FPGA} is connected to its own external \gls{SRAM} memory.

\subsection{\gls{SCU}}

The \gls{SCU}, or System Control Unit, is an Energy Micro EFM32GG390F1024-BGA112 microcontroller which administrates communication between the \gls{FPGA} and the outside world.
It is the \gls{SCU}'s job to react to user input, to program the custom processor implemented on the \gls{FPGA} and to perform other administrative duties.
Having a separate microcontroller to perform these tasks minimizes the complexity of the design implemented on the \gls{FPGA}, which means that more resources can be used to create a performant and clean custom processor design.

While only one input and one output are strictly needed to meet the functional requirements, multiple different types of input have been added to the design in the name of safety through redundancy.
The different inputs and outputs are listed in the following subsections.

\subsubsection{USB}

Universal Serial Bus is a bus interface used to facilitate the communication between Barricelli and a host computer.
USB support has become very widespread, to the point where it is difficult to find a computer without several USB ports, and was a natural choice for a main communication link.
Another important reason for choosing USB was because of the large number of software libraries and examples available.

\subsubsection{TIA-232}

TIA-232, more commonly known as RS-232, is a set of standards defining communication over what is commonly known as serial ports.
The serial link is an old standard and a comparatively slow way to transfer data, so it is mainly included as a fallback in case the USB link should fail.

\subsubsection{SD}

Secure Digital (SD) is a memory-card format chosen as a secondary backup as communications channel between the developers and the microcontroller.
Should both the USB and serial ports fail, an SD card could be loaded with the desired instruction and data memory and be read by the microcontroller as if it was a bus.

\subsubsection{LEDs}

The simplest way to get output from the microcontroller is to use LEDs, short for Light-Emitting Diode.
There are a total of 16 LEDs which can be used to display status from the IO controller, or simply show that the IO controller is working on something.

\subsubsection{Buttons}

Buttons is the fastest and simplest way of having some form of user input on the board.
The buttons allow the user to interact with the program running on the SCU without requiring some external IO.

\subsection{Memory}

The \Gls{galapagos} CPU is connected to two separate external memories, the instruction memory and the data memory.
This separation means that the \Gls{galapagos} architecture is a Harvard machine.
This design choice was made because it increases the performance of the machine, since instruction memory and data memory can be accessed independently, and in parallel. 

\subsubsection{Instruction Memory}

The Instruction Memory, labeled as ``INST MEM'' in figure \vref{figure:system-overview}, is the memory that holds the program instructions for the processing units in the \Gls{barricelli} running on the \gls{FPGA}.
The memory is read-only for the processors in the \gls{FPGA}, but can be written to by the \gls{SCU}.
The instruction memory is shared by all the processing units in the \gls{FPGA}.

\subsubsection{Data Memory}

The Data Memory, labeled as ``DATA MEM'' in figure \vref{figure:system-overview}, is the memory that holds processing data for the processors in the \gls{FPGA}.
The memory can be read by both the processing units in the \gls{FPGA}, and the \gls{SCU}.
The data memory is shared by all the processing units in the \gls{FPGA}.
