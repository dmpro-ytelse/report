\section{System Architecture}

This section aims to provide a birds-eye overview of the system archictecture of the \Gls{barricelli} computer.

\begin{figure}[H]
\includegraphics[width=\textwidth]{dmpro/Sketch 1 - System Overview.png}
\caption{System Overview}
\label{figure:system-overview}
\end{figure}

The requirements stated\cn that the \Gls{barricelli} should use an FPGA and an micro controller as system components.
The \Gls{barricelli}'s specialized CPU design is implemented and compiled onto an FPGA, and a micro controller is used to facilitate I/O, control and communications between the FPGA and the outside world.
A graphical overview illustrating the system architecture can be found in figure \vref{figure:system-overview}.

\section{Components}

This section contains an exhaustive list of the components in figure \vref{figure:system-overview}, documenting their roles in the system architecture.

\subsection{FPGA}

\todo{What is it, what does it do in the system?}
The FPGA is a Xilinx Spartan6 LX45\cn Field-Programmable Gate Array, which is a microchip which can be programmed to behave like any arbitrary intergrated circuit.
The FPGA is programmed to contain an implementation of the custom \Gls{galapagos} CPU Architecture designed for the \Gls{barricelli} computer.
It is connected to the SCU by a 39-bit\cn wide bus, though which the CPU can be programmed by users.
The FPGA is connected to its own memory.

\subsection{SCU}

The SCU, or System Control Unit, is an Energy Micro EFM32GG990 micro controller which administrates communication between the FPGA and the outside world.
It is the SCU's job to react to user input, to program the custom CPU implemented on the FPGA and to perform other administrative duties.
Having a separate micro controller to perform these tasks minimises the complexity of the design implemented on the FPGA, which means that more resources can be used to create a performant and clean custom CPU design.

\subsection{I/O}

The Barricelli has several different inputs and outputs.
While only one input and one output are strictly needed for a functional system, multiple different types of input have been added to the design in the name of safety through redendancy.
Having multiple communication channels between the user and the system means that even if one or more channels should fail, the system will still be operational.

\subsubsection{TIA-232}

TIA-232, otherwise known as RS-232, is a standard describing transfer of data using a serial cable. \todo{more, cite RS-232 or TIA-232}

To easily communicate with the microcontroller, a serial port was added to the schematics. The micro controller used in the project has a built in UART device\cite{efm32gg990-datasheet}, 
and to use it in the project is easily done. As it is a slow way to transfer data, it will be used more as a fallback to the USB.

\subsubsection{USB}

Universal Serial Bus is a bus interface.
\todo{What is it, what does it do in the system?}

\subsubsection{SD}

\todo{What is it, what does it do in the system?}

\subsubsection{LEDs}

\todo{What is it, what does it do in the system?}

\subsubsection{Buttons}

\todo{What is it, what does it do in the system?}

\subsection{Memory}

The \Gls{galapagos} CPU is connected to two separate external memories, the instruction memory and the data memory, meaning that the \Gls{galapagos} architecture is a Harvard machine.
This design choice was made because it increases the performance of the machine, since instruction memory and data memory and be accessed independantly, and in parallel. 

\subsubsection{Instruction Memory}

The Instruction Memory, labeled as ``INST MEM'' in figure \vref{figure:system-overview}, ...
\todo{What is it, what does it do in the system?}

\subsubsection{Data Memory}

The Data Memory, labeled as ``DATA MEM'' in figure \vref{figure:system-overview}, ...

\todo{What is it, what does it do in the system?}
