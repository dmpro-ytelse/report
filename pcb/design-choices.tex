\section{Design choices} 
\label{pcb:design-choices}
This section highlights and explains the choices made relating to the hardware of the final PCB.

\subsection{Field Programmable Gate Array (FPGA)} \label{pcb:design-choices:ss:fpga}
Per the project's second non-functional requirement\vref{table:non-functional-requirements}, the system's FPGA was required to be one produced by Xilinx.
The Spartan-6 family was chosen because development boards with the Spartan-6 LX16 FPGA was available at the lab.
Higher-numbered Spartan-6 FPGAs have more resources but fewer I/O ports and a higher price tag than lower-numbered ones.

A Xilinx Spartan-6 LX45 (specifically, the XC6SLX45-2CSG324I) was chosen because of it having a sufficient number of resources and enough I/O ports while sporting a reasonable price tag.

\subsection{Microcontroller / System Control Unit (SCU)} \label{pcb:design-choices:ss:scu}
The EFM32 Giant Gecko 32-bit Microcontroller, was chosen for the project.
This microcontroller fulfills the non-functional requirement of using a microcontroller made by Energy Micro\vref{table:non-functional-requirements}.
In addition there were development boards for the Giant Gecko 32-bit microcontroller available at the lab.

The EFM32GG390F1024-BGA112 was chosen because it was deemed powerful enough to satisfy the performance requirements and had the highest number of available general purpose I/O (GPIO) pins amongst the microcontrollers with the same kind of package.

\subsection{Communication} \label{pcb:design-choices:ss:internal_communication}
The two major components on the system are the SCU and the FPGA.
They each fill an important role, and work on essential tasks.
The system needs them both to work together, and to accomplish this a communication channel is needed.
On Barricelli this was a 41 bit bus.
This bus has 16 bits of data, 19 bits of addressing, a small 2 bit bus to control the state of the processor and 3 control signals.
The stated bus determine what the FPGA should do with the data, while the control signals tells the targeted unit what to do with the data. 


\subsection{Input/Output devices} \label{pcb:design-choices:ss:IO_devices}
This section presents the Input/Output (I/O) devices that were selected and discusses alternatives that were not.
An I/O device or channel that can be used to communicate between a computer system and the outside world (or another computer system).

\subsubsection{Secure Digital (SD)} 
The Secure Digital (SD) memory card format was chosen because there were SD cards available in the lab and most of the team's members' laptops had SD card slots.
The microSD format was considered, but guidelines on how to use and implement it were scarce compared to information about the larger SD format.

In the SD interface, there are several protocols used for communication. 
The ``Serial Peripheral Interface Bus'' transfer mode (``SPI bus mode'') was chosen for the project as it allows the microcontroller to communicate with the SD card as if it were a bus.


\subsubsection{Universal Serial Bus (USB)}
The USB interface was chosen because the chosen SCU has a built-in USB driver, which reduced the amount of work required to implement the standard considerably.
USB connectors are prevalent on computers, and every team member's laptop had at least one USB connector
A micro-USB interface was chosen because it was the smallest USB compliant interface available, which meant the associated hardware would take up less physical space on the PCB than its larger siblings'.

The design also includes circuits to prevent undesirable effects like electrostatic discharge, preventing the signals from picking up unwanted background noise, and crosstalk (disturbance of the signal from signals in other circuits).
Resistors were also added to prevent short circuiting.

There are several configurations applicable to the USB interface.
The USB protocol specifies that there should be one host (or ``master'') and at least one or more ``slaves''.
The master is responsible for managing the connection to its slaves, and should also be able to provide a 5V current to its slaves if needed.

In Barricelli's case, the microcontroller functions as a self-powered slave.
The device that is connected through the USB interface in order to communicate with the microcontroller functions as the master.

The circuit design was copied from the microcontroller's developer's application notes~\cite[Figure 2.2]{an0046}.

\subsubsection{Ethernet}
Ethernet support was not added to the system in favor of USB the chosen microcontroller did not have built-in support for Ethernet.


\subsubsection{Serial Port/RS-232} 
RS-232 (colloquially known as a ``serial port'') is a communication standard which is implemented through a serial port interface.
Even though serial communication was not required due to presence of USB interface, the decision was made to implement it.
The serial port serves as a backup for the USB as the chosen microcontroller supports communication over RS-232.

\subsection{Memory} \label{pcb:design-choices:ss:memory}
The FPGA group wanted memory that would provide fast accesses in combination with a large input/output capacity in order maximize speed and efficiency.
SRAM (Static random-access memory) chips were selected selected because of its fast access times.
Reasonably priced 8 Mbit (512K x 16 bit) modules with $10$ns write and read cycle times were found to be suitable and selected. 

\subsection{Crystal}
A $32.768$KHz crystal was selected to drive the microcontroller.
However as the project progressed and the microcontroller's application notes were read more closely it was discovered that it did not need an external crystal to drive it as it had an internal clock.
In addition the $32.768$KHz crystal was connected to the wrong ports. 
