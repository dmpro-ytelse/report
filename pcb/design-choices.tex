\section {Design choices}

Here we will talk about the choices that were made that largly affected what the final PCB were like and what hardware it contained.

\subsection{IO devices} \label{pcb:design-choices:ss:IO_devices}
Here we will talk about the IO devices we picked and about the other alternatives we did not choose..

\subsubsection{SD}
We talked about using micro-sd cards, but because it was harder to find any guidlines on how to use/implement it we decided to use the normal SD interface.
In the SD interface, there are several protocols used for communication. 
However for this project we use the mode called "SPI bus mode", which is the interface that is used by the microcontroller.  --verify IO group
The SPI or "serial peripheral interface" is a scheme where you use a syncronized master-slave configuration or relation to communicate with the IO-devices. 
The goal with this choice is to have better compatibillity with the microcontroller.

\subsubsection{USB}
For usb interface we chosed the micro-usb. 
The reason for this is that the size of the assosiated hardware is much smaller.
In the design we also made circuts to prevent unwanted effects like electrostatic discharge, and circuts to prevent the signals from picking up unwanted noice from the background or from crosstalk.
Of course we also added neccessary resistances to prevent shortcircuting.


\subsubsection{RS-232}
Even though serial communication was technically not required due to presence of USB interface, it was decided to implement RS-232 communication as well.
This way in case there were troubles with USB, it was still possible to communicate over RS-232; and vice versa.
The microcontroller supports communication over RS-232, so there were no problems there.
We placed socket connector on the PCB as well as a RS-232 driver that transforms input and output signals to given voltage.

\subsection{Communication} \label{pcb:design-choices:ss:internal_communication}

Here we will talk about how our internal commmunication on the board work, and how it was planed.

\subsection{Memory} \label{pcb:design-choices:ss:memory}

Here we will talk about the memory we have, why it was picked and what other alternatives we did not choose.
