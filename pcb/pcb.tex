\begin{figure}[H]
\centering
\includegraphics[width=10cm,keepaspectratio]{pcb/thewholeboard.jpg}
\caption{The final design of the PCB.  }
\label{figure:thewholeboard}
\end{figure}
A lot of effort has been invested to make the board as small as possible under the design process in order to make the board fit a small cabinet that was made by the group for the project.
While the various cores of the genetics pipeline are the central components of the Barricelli system, the system itself would be little more than a simulation on some developer board without the printed circuit board (PCB) connecting all the components.
Designing the PCB and soldering components onto it are therefore important aspects of the development process of the Barricelli system.

In this chapter the PCB is presented.
The design and production processes are detailed.
Explanations are provided for why certain components were chosen.
And encountered problems and their workarounds are presented.

\section {Design choices}
This section highlights and explains the choices made relating to the hardware of the final PCB.

\subsection{Input/Output devices} \label{pcb:design-choices:ss:IO_devices}
This section presents the Input/Output (I/O) devices that were selected and discuss alternatives that were not.
An I/O device or channel that can be used to communicate between a computer system and the outside world (or another computer system).

\subsubsection{Secure Digital (SD)} 
The Secure Digital (SD) memory card format was chosen because there were SD cards available in the lab and most of the team's members' laptops had SD card slots.
The microSD format was considered, but guidelines on how to use and implement it were scarce compared to information about the larger SD format.

In the SD interface, there are several protocols used for communication. 
The ``Serial Peripheral Interface Bus'' transfer mode (``SPI bus mode'') was chosen for the project as it allows the microcontroller to communicate with the SD card as if it were a bus.


\subsubsection{Universal Serial Bus (USB)}
The USB interface was chosen because USB connectors are prevalent on computers, and every team member's laptop had at least one USB connector.
A micro-USB interface was chosen because it was the smallest USB compliant interface available, which meant the associated hardware would take up less physical space on the PCB than its larger siblings'.

The design also includes circuits to prevent undesirable effects like electrostatic discharge, preventing the signals from picking up unwanted background noise, and crosstalk (disturbance of the signal from signals in other circuits).
Resistors were also added to prevent short circuiting.

There are several configurations applicable to the USB interface.
The USB protocol specifies that there should be one host (or ``master'') and at least one or more ``slaves''.
The master is responsible for managing the connection to its slaves, and should also be able to provide a 5V current to its slaves if needed.

In Barricelli's case, the microcontroller functions as a self-powered slave.
The device that is connected through the USB interface in order to communicate with the microcontroller functions as the master.


\todo{insert picture from http://www.silabs.com/Support\%20Documents/TechnicalDocs/AN0046.pdf}

\subsubsection{serial port} 
\todo{the name rs232 is too ambigious. We need to explain what it is WAIT HAS THIS BEEN DONE ALREADY?}
\todo{also known as UART}

RS-232 is a communication standard/protocol which is implemented through a serial port interface.
Even though serial communication was technically not required due to presence of USB interface, it was decided to implement RS-232 communication as well.
This way, in case there were troubles with USB, it would still be possible to communicate over RS-232; and vice versa.
The microcontroller supports communication over RS-232, so the this should be a natural second choice.
We placed socket connector on the PCB as well as a RS-232 driver that transforms input and output signals to given voltage.
UART: UART or Serial port was primarily selected as a backup choice for the USB interface. This was made only in case the implementation
of the USB goes wrong either by a bad PCB design or if there by some reason is problems related to software that would cause the USB interface to not work as intended.


\subsection{FPGA} \label{pcb:design-choices:ss:fpga}
\todo{which one did we pick?}

\todo{why did we pick this one?}

\subsection{SCU / Microcontroller} \label{pcb:design-choices:ss:scu}
\todo{which one did we pick?}
we got the giant gecko one because we had the dev board for those lying around in the lab and also we picked the model with the most memory? \todo{VERIFICATION REQUIRED}.

\todo{why did we pick this one?}

\subsection{Communication} \label{pcb:design-choices:ss:internal_communication}

The two major components on the system are the SCU and the FPGA.
They each fill an important role, and work on essential tasks.
The system need them both to work together, and to accomplish this we need a communication channel.
On Barricelli this was accomplished with by a 41 bit bus.
This bus have 16 bits of data, 19 bits of addressing, a small 2 bit bus to control the state of the processor and 3 control signals.
The statet bus detemine what the FPGA should do with the data, while the control signals tells the targeted unit what to do with the data. 



\subsection{Memory} \label{pcb:design-choices:ss:memory}
When the group talked about selecting memory, there was lots of options.
The FPGA group needed a memory that could provide fast accesses in order to increase efficiency.
In addition a memory with large inout/output size was a goal.
Because of this, the group decided to use SRAM because of the fast access times.
However, the downside with these memories is that they are small and expensive.
Still we ended up with 8-MBit modules with 16bit input/output size, 10ns write cycle time and 10ns read cycle time.
A larger size of the memory could probably have been chosen, without any risk of going over the budget.
However for the task that the FPGA is going to perform the current selection is sufficient for our use.

 \label{pcb:section:design_choices}

\section {Power supply}

%% Add a picture of the power supply

As our requirements for the power supply were quite simmilar to the requirements of earlier projects from the subject.
The power supply from the Festiva Lente system was reused in our system.
This power supply have been used for many years, with small changes improving the behavior and performance of the power supply.
To avoid introducing new problems, reusing this power supply was a safe choice.
The barricelli system does however not require any 2.5 volt or 5 volt power.
As a result of this, these parts of the power supply have been removed in our system, and only 12 volt, 3.3 volt and 1.2 volt power is available in our system.

-insert picture here?

 \label{pcb:section:power_supply}

\section {Power plane}



For simpler routing, reduced noise and to reduce voltage drop we have used power nets in this project.
As shown in above the PCB have a dedicated layer for power.
There is a wide track with 1.2 volts for the FPGA and the rest of the layer is one large 3.3 volt power net for all the other components.
In addition to these nets there is a dedicated layer for ground. The reason why we selected the design is done in this way is to provide as short routing path as possible for the sources using the power planes.
Keeping a short distance as possible on all signals is important in order to ensure as low loss of effect (measured in Watts) as possible.

\begin{figure}
\centering
\includegraphics[width=10cm,keepaspectratio]{pcb/powerplanephoto.png}
\caption{Final design of the power plane. The power plane for 1.2V is the long polygon that goes from the ``powersupply'' part of the board to the FPGA core. The
design is also focusing on having as low path as possible to all the components that are using the power grid.  }
\label{figure:powerplanephoto}
\end{figure} \label{pcb:section:power_plane}

\section{Footprints} \label{pcb:footprints}
This section details how various footprints for components used in the project were obtained.

\subsection{Obtaining a footprint}
Once a type of component has been decided upon for a project a specific instance of said component must be decided upon.
In order for said component to fit onto the PCB and properly function, its footprint, must be placed somewhere on the PCB.
The footprint is a kind of blueprint containing a component's outline and pads.

Obtaining a footprint for a component typically involves creating one manually or using a wizard based on the information contained in the component's datasheet.
Some manufacturers make footprints for their components available on their websites, however they might not be available in a format that is understandable by whatever PCB design suite that is employed in the current project.
Altium Designer (version 13.3) feature a browsable database of footprints for various components, all of which can be used immediately in any Altium project.

If a component's footprint is not readily available it has to be created manually.
The most important aspect of this process is to obtain the component's technical datasheet and examine it for a description of the component's package and dimensions.
This is sometimes labeled as an outline drawing, suggested land pattern or package outline.
It is important to notice what system the supplied measurements are in, as mixing for example imperial and metric units in a project could lead to unforseen incompatabilities.
Once one gets a hang of Altium PCB editor it takes surprisingly little time to create a footprint.

For components with standarized packages, Altium has an IPC compliant footprint wizard that generates footprints for a component given its package type and some package specific measurements available in the component's datasheet.


We encountred one issue while designing footprints.
The pads of footprint that was designed for polarized capasitor component EEE1HA1R0AR matched the size of components pins (they were in fact even smaller - 4.7mm vs max 5.5mm in specefication). 
\todo{insert figure? ref figure? figure pol\_cap\_size\_footprint\_vs\_specs.png}
This made soldering nontrivial.
\todo{figure pol\_cap\_size\_on\_board.png}
To avoid this complication the pads on footprint should be designed to be longer than the pins.
That will make the pads protrude under the actual component resulting in much simpler soldering.

USB:

the footprint for the micro-usb was found in the altium libary. 


\todo{as we can see in the picture, the usb have 4 pins that it uses. pin 1 is where the signal goes and.....}

SD-CARD:

The footprint for the SD-card was designed by the group. The reason for this was that appearently the sd-card interface is not standardized, so that 
every manufacturer have their own design on the receptacles. 


\todo{FIND THIS PICTURE: the picture shows the interface of the SD card. the different pins that can be seen on the picture are a part of the standardized SD-protocol that every manufacturer have to follow...}

 \label{pcb:section:footprints}

\section{Budget}
The project's budget was $10 000$ NOK~\cite{assignment-text}.
The Computer Design project's budget was $23000$ NOK per group in 2010 and 2011, and was reduced to $10 000$ NOK per group in 2012\cite{previous-projects} while the price of a ham and cheese sandwich from SiT Storkiosk at Gløshaugen has increased from $29$ NOK in 2010 to $43$ in 2013.
Markets are weird like that.

An non-negotionable criteria for all components selected (excepting the FPGA and microcontroller) was that Farnell had enough of it in stock in their UK storage.
Generic surface mounted devices (resistors and capacitors) were required to fit the 1206 or 1210 package so that the same footprint could be used for all of them.

Components were required to function at $3.3$V or $1.2$V.
If two seemingly equal components passed the aformentioned requirements, the cheapest one was selected.

The minimum amount of required components to produce one functioning Barricelli system cost $1240$ NOK.
However, some components could not be ordered in increments of one.
Taking this into account, the price of one functioning Barricelli system was $1472$ NOK.

The PCB cost $7795$ NOK to produce.
In total, the price of producing a single Barricelli system is $9267$ NOK -- $733$ below budget.

Approximately $300$ NOK of the leftover money was used to construct the 3D-printed case for Barricelli.


 \label{pcb:section:budget}

\section {Process}

Here we will introduse the design process.

\subsection{System design} \label{pcb:process:ss:system_design}

Here we will talk about how the system itself is designed on a logical level.

some write ideas:
--had to make several footprint ourselves:SD card
--no back ground knowledge at all
--hard to verify/test design

\subsection{PCB design and routing} \label{pcb:process:ss:pcb_design_and_soldering}

Here we will talk about how the design was transfered to the PCB, what problems were discovered in this process and how they were solved.

In the planning phase, the group estimated that the design of the PCB and the routing process using the autorouter, would take about 3-4 days. However in reality it took much longer then estimated. The reason for this was that several
problems were encountered during the designphase and routing of the PCB. After we used the autorouter once, we found out that the autorouter violated several design constraints for the board. It was then tried to reroute the board several times
with different options in an attempt to fix the problem. Since the auto routing process took about six hours on the lab computers, the group decided to use some powerful private computers to do the routing. This reduced the time it takes for autorouting from about 6 hours to 1.

After some attempts on the autorouter, the group decided to manually route the last signals in addition to fix the constraint violation. This was a time consuming process, but during this, several serious design flaws were uncovered which would have costed
more time due to the need to produce new boards. Among the errors that were discovered, was that the footprint for the microcontroller was wrong. Several of the capacitators were also unconnected, or just connected wrong. This means that even though the manual routing of the PCB took longer,
the discovery of the designflaws probably saved us some time in total.
\subsection{Soldering} \label{pcb:process:ss:soldering}

Here we will talk about the soldering process and how we worked with that.
This will not cover major problems that needed a workaround (They are covered in their own chapter), but rather the challenges we experienced in the soldering process.

-had to coordinate everything and work on shifts in order to get thigns doen
-got a plant instead of the microcontroller in mail
-should have used altium for component lists
 \label{pcb:section:process}

\section {Problems and workaround}
Here we will talk about various problems we discovered during the soldering process, and how we found ways to workaround them.
We expect that there will be some things that may be possible to work around in code on the microcontroller, and other parts that require hardware fixes.
There might also be problems that cause parts of the board to not function.

\subsection{ Power connector footprint }

The footprint of the power connector had three pairs of holes instead of a milled groove.
This caused the connector to not fit in the footprint.
This was however solved by cutting away the parts of the connector that did not fit on the PCB using pliers.
The result worked fine, and it's hard to spot that the power connector is modified if you do not have a correctly mounted connector as a reference.

\subsection{ FPGA to SCU bus routing }

Because of an error made during the routing of the board, the header pin for FPGA\_ENABLE is not connected to any FPGA pins.
This error can be corrected by using one of our spare FPGA lines available on headers.
A wire was pulled from FPGA\_HEADER78 to the header from the SCU, and this header cable allowed us to run the rest of the bus as planed.

\subsection{ USB port }
\label{subsec:pcb:problem:usb}
\begin{figure}
\centering
\includegraphics[width=10cm,keepaspectratio]{pcb/vreghack.jpg}
\caption{The figure shows the ``hack'' that were made on the PCB in an attempt to fix the USB. Sadly the board were accidentally shortcircuted and died before this the hack could be fully verified to be working}
\label{figure:vreghack}
\end{figure}

When the PCB came back from production, it was discovered that the USB was not connected to the microcontroller
according to the recommended specifications from the manufacturer of the microcontroller. 
The problem was that the signals USB\_VBUS and USB\_VREGI were not connected to the VBUS-pin on the USB receptacle. This was fixed by soldering copper wires on the capacitors designated as C18 and C17 to USB-HEADER2 (called VBUS\_ENABLE in the schematics).
According to tests performed on the PCB before and after this work-around, the problem was fixed successfully.

\todo{write about the oscillator?}

\todo{write about the crystal -- this has been mentioned in another section already so just copy paste based on that}
 \label{pcb:section:problems_and_workaround}

