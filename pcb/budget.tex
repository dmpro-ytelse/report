\section{Budget}
The project's budget was $10 000$ NOK~\cite{assignment-text}.
The Computer Design project's budget was $23000$ NOK per group in 2010 and 2011, and was reduced to $10 000$ NOK per group in 2012\cite{}\todo{cite http://www.idi.ntnu.no/emner/tdt4295/oldproj} while the price of a ham and cheese sandwich from SiT Storkiosk at Gløshaugen has increased from $29$ NOK in 2010 to $43$ in 2013.
Markets are weird like that.

An non-negotionable criteria for all components selected (excepting the FPGA and microcontroller) was that Farnell had enough of it in stock in their UK storage.
Generic surface mounted devices (resistors and capacitors) were required to fit the 1206 or 1210 package so that the same footprint could be used for all of them.

Components were required to function at $3.3$V or $1.2$V.
If two seemingly equal components passed the aformentioned requirements, the cheapest one was selected.

The minimum amount of required components to produce one functioning Barricelli system cost $1240$ NOK.
However, some components could not be ordered in increments of one.
Taking this into account, the price of one functioning Barricelli system was $1472$ NOK.

The PCB cost $7795$ NOK to produce.
In total, the price of producing a single Barricelli system is $9267$ NOK -- $733$ below budget.

Approximately $300$ NOK of the leftover money was used to construct the 3D-printed case for Barricelli.


