\section {Components}

Here we will talk about the main components that were seleted for the PCB.
This will include everything exept trivial components like resistors and capasitors, as they are mostly selected because of their values and does not affect the functionality of the board.

Memory: 

Microcontroller: we got the giant gecko one because we had the dev board for those lying around in the lab and also we picked the model with the most memory? \todo{VERIFICATION REQUIRED}.

FPGA: wait why did we pick this FPGA? \todo{why did we pick the FPGA we picked?}

USB: As mentioned in 5.1.1, a micro usb receptacle was selected to be placed on the board. The usb device was selected in order
to fulfill functional requirement 5, "The system should be able to send data to an
external entity". Although it is possible to communicate with the FPGA through the JTAG pins (Debug pins),
the group wanted a good standarized interface to an external computer in order to save time and space on wireings.

SD: THe SD card slot was made in order to fulfill requirement 4, "The system should be able to send data to an
external entity". Also this was also selected in order to have the posibility to use the flash memory on the SD card as memory for the processor. 
The only downside with the SD card is that the interface is relatively slow, which makes it lesser favourable for using it as memory in our case.


UART: UART or Serial port was primarily selected as a backup choice for the USB interface. This was made only in case the implementation
of the USB goes wrong either by a bad PCB design or if there by some reason is problems related to software that would cause the USB interface to not work as intended.

\todo{figure out what components should be mentioned here}
