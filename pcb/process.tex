\section {Process}

Here we will introduce the design process.

\subsection{System design} \label{pcb:process:ss:system_design}

Here we will talk about how the system itself is designed on a logical level.

some write ideas:
--had to make several footprint ourselves:SD card
--no back ground knowledge at all
--hard to verify/test design

\subsection{PCB design and routing} \label{pcb:process:ss:pcb_design_and_soldering}

Here we will talk about how the design was transferred to the PCB, what problems were discovered in this process and how they were solved.

In the planning phase, the group estimated that the design of the PCB and the routing process using the auto-router, would take about 3-4 days. However in reality it took much longer then estimated. The reason for this was that several
problems were encountered during the design phase and routing of the PCB. After the first auto-routing run, it was discovered that the auto-router violated several design constraints for the board. It was then tried to reroute the board several times
with different options in an attempt to fix the problem. Since the auto routing process took about six hours on the lab computers, the group decided to use more powerful private computers to do the routing. This reduced the time it takes for auto-routing from about 6 hours to 1.

After some attempts on the auto-router, the group decided to manually route the last signals in addition to fix the constraint violations. This was a time consuming process, but during this, several serious design flaws were uncovered which would have costed
more time due to the need to produce new boards. Among the errors that were discovered, was that the footprint for the microcontroller was wrong. Several of the capacitors were also unconnected, or just connected in a wrong way. This means that even though the manual routing of the PCB took longer,
the discovery of the design flaws probably saved us some time in total.
\subsection{Soldering} \label{pcb:process:ss:soldering}

Here we will talk about the soldering process and how we worked with that.
This will not cover major problems that needed a workaround (They are covered in their own chapter), but rather the challenges we experienced in the soldering process.

Due to numerous delays in the design of the schematics and the routing of the PCB, the group had to complete the soldering process as fast as possible.
In order to do this effectively, the group coordinated the work in shifts so that people were working on soldering the PCB both day and night time.

There were also some problems encountered during the soldering process. The most significant problem was that it was discovered that
we received voltage regulators instead of the microcontrollers that were needed. This caused some delay to the soldering process.

Also the ordering of the needed components were also done in many turns instead of one large, single order.
This happened because the components list were not always updated when the schematics changed. The result of this were
that too few components were ordered. First after all the components that was needed were ordered, it was discovered that there was functionality in Altium that could be used
to generate component lists. Using Altium instead of manually updating the component list could probably have saved some time.

