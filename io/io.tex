\todo{Microcontroller (efm32gg990f1024) handling I/O between computer, buttons, LEDs and FPGA. Redundancy, can manage with one of SD/USB/Serial.}

\section{IO Channels/hardware}
\todo{Subsectionise, detailed about everything, including software/libraries used to facilitate communication.}
\begin{itemize}
    \item Data IO
    \begin{itemize}
        \item SD
        \item USB
        \item Serial
    \end{itemize}
    \item LEDs, buttons
    \item Huge FPGA bus
    \item Upload/debug connection
\end{itemize}

\section{FPGA Control}
\todo{Direct memory access, only way of talking with FPGA. A few diagrams, timing etc.}

\section{Interaction}
\todo{Main loop, state machine (diagrams ++).}
\subsection{Host program}
\todo{Basically the same as above, but for the computer talking with the FPGA.}

\section{Design decision}
\begin{itemize}
    \item The microcontroller chosen is the same as in the available development kit. As there was no restrictions on power usage, 
          it was decided to use the micro controller most convenient to develop on.
    \item Early in the process, a discussion arised about how it could be beneficial to run an operating system on the micro controller
          such that familiar programs could be used on it. A scenario pitched was to have network access, and then be able to talk to the
          machine remotely using programs such as \textit{SSH} or \textit{telnet}. However, as the Linux version available on the Energy Micro
          system was deemed not so useful, and the controller misses network support, it was decided that running an OS was unnecessarily much work.
    \item FPGA Bus, huge
\end{itemize}

\section{Issues?}
\begin{itemize}
    \item Not getting work done
    \item Crystal
    \item USB VBus
\end{itemize}
