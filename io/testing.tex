\subsection{IO device tests} \label{iotest}
\test
{Buttons \& LEDs}{
    \item{Upload a program reading the state of all buttons and turning off the LEDs corresponding to the buttons pressed down while leaving the rest of the LEDs on}
    \item{Try pressing the different buttons}
}{All LEDs light up initially and turn off when the corresponding button is pressed.}
{All LEDs light up initially and turn off when the corresponding button is pressed.}

\test
{Debug connection test}{
    \item{Connect the debug pins to the appropriate pins on the energy micro development kit}
    \item{Turn the debug to OUT}
    \item{Connect to the development kit using energyAware commander}
}{EFM32GG990F1024 listed as microcontroller}
{EFM32GG990F1024 listed as microcontroller}

\test
{SD Card test}{
    \item{Edit the code from AN0030~\cite{an0030} to use correct pins}
    \item{Compile the code, upload and run it, with SD Card connected}
    \item{Confirm data on SD Card}
}{File with string ``EFM32 ...the world's most energy friendly microcontrollers !'' is added to the SD Card.}
{The SD card was not found, and the text not present on the SD Card}

\test
{USB test}{
    \item{Compile the code from AN0065~\cite{an0065}}
    \item{Upload and run it}
    \item{Run the supplied host PC program while connected to the PCB through USB}
}{The host programs runs successfully}
{The host program fails to connect through USB}

\test
{Serial test}{
    \item{Compile the code from AN0045~\cite{an0045}}
    \item{Upload and run it}
    \item{Connect the host PC to the PCB and run terminal emulator of choice}
}{"Energy Micro RS-232 - Please press a key" appears in the terminal}
{No output in terminal}

\subsection{FPGA bus}
\test
{SRAM test}{
    \item{Write a value to a range of addresses}
    \item{Read the same address and compare with the value written}
}{The values are identical}
{Most of the values are identical, with some addresses reporting the wrong value}

\test
{Running a program}{
    \item{Upload a program to fill the memory with fibonacci numbers}
    \item{Let the CPU run for a while to ensure that somehting has been written to memory.}
    \item{Read memory and check whether the fibonacci numbers are stored, the first on adress 0, the next on the next address and so on.}
}{A sequence of fibonacci numbers in the memory.}
{Seemingly random data read from the memory}
