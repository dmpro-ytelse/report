We designed and tested a system capable of solving hard problems using genetic algorithms.
That is, the system's PCB design looks sound on paper and the processor has been successfully tested with simulations.
However we were unable to deliver a functioning physically integrated version of the system because the ones we produced were eventually bricked for various reasons.
Even so, we still fulfilled all our functional and non-functional requirements because as it turns out none of these require a working, physically integrated\footnote{with all the components soldered on} system.

Our goal of constructing a general MIMD computer capable of solving hard problems using genetic algorithms was met.

It is capable of receiving instructions and data from external entities through the debug pins.
The USB, SD and serial ports were not successfully tested due to the problems with the PCB.

We implemented performance increasing techniques that would not overly complicate the design.

The Galapagos instruction set includes instructions to load and store genes from and to a genetics pool, which have eased the task of working with genetic algorithms with the instruction set.

Soldering all the required components -- including the Xilinx FPGA and ``Energy Micro'' (now Silicon Labs) microcontroller --  onto the system took approximately eight hours.
Although only one completely soldered board was produced: the time required to solder a board would likely have been reduced with each further board produced.

The cost of producing one complete system was below budget.
The PCB's production cost ate up $78\%$ of the budget.

The system's 3D-printed case is pretty neat and a complete Barricelli system can fit inside of it. 
It has all the visual features of the barricelli board, including access to the I/O ports, I/O buttons and LEDs, reset and power indicator.

And of course there's a report: the very same that you are reading right now.

The project has been challenging.
It feels somewhat bizarre that we got so far.
They say experience is its own reward, if that is true we've been rewarded plenty.

