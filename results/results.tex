This section section describes the results of different measurements, calculations and tests that were run on the Barricelli computer.
It also documents the different demonstration programs that showcase and illustrate Barricelli's purpose through practical use.

\section{Measurements}

\subsection{Performance}

\todo{measure performance with different core amounts}

\todo{measure performance with and without genetics pipeline}

\todo{measure performance of same problems on consumer grade laptop}

\todo{measure performance of general program}

\subsection{Energy Efficiency}

\todo{measure power usage (idle, working, etc)}

\section{Demonstration Programs}

This section documents the demonstration programs written for the Barricelli computer to demonstrate its functionality.
The programs are typically written in \gls{galapagos} assembly for programs runnin on the custom processor, and C for programs running on the \Gls{SCU}.
The source code for these demonstration programs can be found in appendix \vref{appendix:demonstration-programs-source-code}.

\subsection{Genetic Algorithm: Color Search}

The color search program is a very simple program demonstrating a basic usage of the genetics accelerator.
The program tries to find a specific color in the search-space of all 24-bit colors.
An individual represents a specific 24-bit color in RGB format.
The individual is coded to a 64-bit data word like in figure \vref{figure:color-search-bytefield}.

\begin{figure}[H]
    \begin{center}
        \begin{bytefield}[bitwidth=0.5em,endianness=big]{64}
            \bitheader[bitformatting={\tiny\rotatebox[origin=B]{90}}]{0, 7, 8, 15, 16, 23, 24, 63} \\
        \bitbox{40}{\color{lightgray}\rule{\width}{\height}}
            \bitbox{8}{blue}
            \bitbox{8}{green}
            \bitbox{8}{red}
        \end{bytefield}
        \caption{The binary coding of an individual for the color search problem}
        \label{figure:color-search-bytefield}
    \end{center}
\end{figure}

Figure \vref{figure:color-search} shows the evolution of the approximation suggested as an answer by the genetic algorithm.
In this problem instance the target color was magic pink, i.e. the color with color code $ rgb(255, 0, 255) $.
The program run illustrated in figure \vref{figure:color-search} ran on 7 seven cores, and the measurements are from regularly polling a single core for its current best solution.
Figure \vref{figure:color-search} clearly illustrates a typical trait of genetic algorithm approximations: they are quite good at finding decent approximations, but iterating to improve accuracy of the result is a game of diminishing returns.
The algorithm quickly finds a decent approximation of the target color, but finding the exact value down to the last bit still takes time.

\begin{figure}[H]
    \begin{center}
        \includegraphics[width=\textwidth]{fig/color-search}
    \caption{Color search progression (7 cores, 1 core sampled)}
    \label{figure:color-search}
    \end{center}
\end{figure}

\subsection{Genetic Algorithm: Binary Knapsack Problem}

\todo{Explain the binary knapsack program}

\subsection{Genetic Algorithm: Travelling Salesman Problem}

\todo{Explain the travelling salesman problem}

\subsection{Blinkenlights}

\todo{Explain the blinkenlights program (it is a program that blinks the lights or something, and maybe reacts to some buttons, to show the IO stuff we have}
