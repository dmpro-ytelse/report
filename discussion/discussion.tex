\section{Performance}
\subsection{performance measurements and benchmarking}
\subsection{average instructions per cycle}
\subsection{performance comparison}

\todo{this}

\section{Energy Efficiency}

\todo{this}
\section{theory}
In this section theory related to MIMD architectures and how to further improve their performances are discussed.
This also covers the discussions inside the group about the design of the architecture in the planning phase of the project.
\subsection{SPMD and concurrency}
One of the first discussions that came up in the group after the assignment were given was if the processor cores should
be able to synchronize themselves. By doing this, the processor cores would be able to execute code that is not completely independent in paralell in an elegant way.
However this also raises some issues as for instance data hazards.

\subsubsection{coherrency}
In addition to data hazards, cache coherency is a large issue....

\subsection{using Cisc or Risc ISAs}
Cisc and Risc instruction set architectures are two very different ways of thinking when it comes to creating instruction sets.
While in the last years, Risc ISAs have been the most dominant instruction set architecture, we can also see that CISC architectures
are on their way back into the markets. Some of the reasons for this is that increasing paralellism is gaining lesser performance increases.

\subsubsection{micro operations: a bridge between Complex and reduced instruction sets}
The use of micro operations is based on the principle that you want to convert a complex instruction into a set of smaller micro operations. This
may simplify the design of for instance a super scalar processor because dependencies between the converted micro instructions would already be known. 


\subsection{Memory management policies}

\section{\todo{probably more stuff}}
