\section{Performance}
\subsection{performance measurements and benchmarking}
\subsection{average instructions per cycle}
\subsection{performance comparison}

\todo{this}

\section{Energy Efficiency}

\todo{this}
\section{theory}
In this section theory related to MIMD architectures and how to further improve their performances are discussed.
This also covers the discussions inside the group about the design of the architecture in the planning phase of the project.
\subsection{SPMD and concurrency}
One of the first discussions that came up in the group after the assignment were given was if the processor cores should
be able to synchronize themselves. By doing this, the processor cores would be able to execute code that is not completely independent in parallel in an elegant way.
However this also raises some issues as for instance data hazards.

\subsubsection{Coherency}
MIMD architecture use a shared memory models. This imposes a problem when using caches, and memory in general. When more core updates on the same values on the same memory positions; memory collisions occur. These problems can be fixed by enforcing that only one core is able to access the memory at any given time. A far more difficult problem is the problem of cache coherence. Cache coherency issues occurs when several cores have private caches containing the same data, and some core changes the data. Then the data in the caches is not consistent among the cores. In order for the data to be consistent, in this example, is for each core having the same data. Note that same data in this context mean data from the same memory location. 

The Galapagos architecture does not support private data caches. This design choice relieves the processor designer of implementing advanced cache coherency algorithms in hardware. Instead of private data caches the Galapagos architectures employ shared pools for rated and un-rated chromosomes. These are connected to a bus and the connected through the \emph{genetic controller}. The controller is configured to only allow one core perform its operation on one of the pools at any given time. This implies that read and write operations are atomic. As a direct consequence cache coherency issues are not possible in the architecture. 
\todo{fix}




\subsection{using CISC or RISC ISAs}
CISC and RISC instruction set architectures are two very different ways of thinking when it comes to creating instruction sets.
While in the last years, RISC ISAs have been the most dominant instruction set architecture, we can also see that CISC architectures
are on their way back into the markets. Some of the reasons for this is that increasing parallelism is gaining lesser performance increases.

\subsubsection{micro operations: a bridge between Complex and reduced instruction sets}
The use of micro operations is based on the principle that you want to convert a complex instruction into a set of smaller micro operations. This
may simplify the design of for instance a super scalar processor because dependencies between the converted micro instructions would already be known. 

\subsection{Memory management policies}
The Galapagos architecture operates with several types of shared memory: \emph{instruction memory}, \emph{data memory}, \emph{instruction caches}, \emph{rated pool} and \emph{unrated pool}. These types of memories  can further be divided into two groups: memory and genetic related. These two groups are connected to separate data and address buses.  \todo{Not sure what to write}


\section{\todo{probably more stuff}}
