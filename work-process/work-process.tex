\section{Development model}
For this project, it was agreed that an adaption of the Scrum development model should be used. The group wanted
to use an agile development method in order to be able to make quick changes to the requirement specification during the project.
The model is adapted however because many of the aspects in scrum, like working in sprints does not work so well for this project due to the unpredictable nature of the project. 
For the same reasons there was no daily meetings, but rather once in a week where everyone is informed about the current progress of the project.
\section{Group Organization}

From early on, the group divided itself into 3 work groups, focusing on three main areas: FPGA, PCB and IO.
This was done largely based on the fact that it has been done similarly in the years before\cite{previous-projects}, and it seemed like a reasonable thing to do.
One advantage in doing this is that each work group can become more specialized in their respective fields compared to if everyone were to work equally on everything.
This allows for a more advanced execution in the project, which is a good thing.
The group member work group allocation can be found in table \vref{table:group-allocation}.

Other than this group allocation, no other hierarchical elements were introduced.
The group functioned as a direct democracy in all other issues, with no appointed leader. This approach had both
it's benefits and downsides. The greatest benefit is that everyone got to take part in the important decisions made in the project.
The downside was that decisions also take much longer time to take due to having to discuss matters over a meeting.


\begin{table}[H]
\begin{center}
\begin{tabular}{l}
\textbf{FPGA} \\
\hline
Sigve Sebastian Farstad \\
Torbjørn Langland \\
Per Thomas Lundal \\
Bjørn Åge Tungesvik \\
\\
\\
\textbf{PCB} \\
\hline
Fedor Fadeev \\
Eirik Flogard \\
Rune Holmgren \\
Odd Magnus Trondrud \\
\\
\\
\textbf{IO} \\
\hline
Emil Taylor Bye \\
Péter Gombos \\
\end{tabular}
\caption{Group Allocation}
\label{table:group-allocation}
\end{center}
\end{table}

The group held weekly sync-up meetings in addition to a weekly status meeting with the course staff.
Unfortunately, meeting attendance was occasionally some lower than what it should have been.
Fortunately, meeting minutes were always kept, so information from missed meetings did not go lost.

\section{Organizational tools}
\subsection{GitHub}
GitHub was used in order for everyone to be able to work at different parts of the project at the same time.
It also provides an excellent version control that would allow a user to work on experimental "branches". When using 
branches, the users does not need to worry about taking backups before trying out something new. These branches can also be merged into the main project later.
\subsection{Trello}
Trello is a tool that the group used as "scrum table" to keep everyone updated in real time about the current progress in the project.
The tool resembles a scrum board which keeps track of what everyone is doing at a specific time.

\section{Tools}
Below here is a list of tools that were used directly to develop the system.

\subsection{Software}
\begin{description}
    \item{\textbf{ISE Project Navigator 12.4 (nt64) M.81d, expired licence}} \\
        Main IDE for writing VHDL.
    \item{\textbf{ISim 12.4 (nt64) M.81d, expired license}} \\
        Main simulation environment for simulating VHDL.
    \item{\textbf{ModelSim SE 6.6d}} \\
        Secondary simulation environment for simulating VHDL.
    \item{\textbf{Xilinx Platform Studio 12.4 (nt64) Build EDK\_MS4.81d+1, expired licence}} \\
        Used for preparing compiled VHDL for the FPGA board.
    \item{\textbf{IAR Embedded Workbench for ARM 6.60.2.5507}} \\
        Used for programming and debugging on the microcontroller.
    \item{\textbf{Avnet Programming Utility}} \\
        Used for configuring the FPGA.
    \item{\textbf{energyAware Commander 2.82}} \\
        Programming and troubleshooting of the microcontroller.
    \item{\textbf{energyAware Designer 1.10}} \\
        Used for configuring the GPIO pins and generating projects for the microcontroller.
    \item{\textbf{Saleae Logic 1.1.9}} \\
        Used to view the sampled waveforms from the logic analyzer
    \item{\textbf{Tera Term Pro Web Version 3.1.3}} \\
        Terminal emulator used for serial communication.
    \item{\textbf{Text editors}} \\
        Sublime Text 2, Vim 7.3, Notepad (©Copyright Microsoft Corporation).
    \item{\textbf{GNU command-line tools}} \\
        Grep, sed, find, etc.
    \item{\textbf{git 1.8.1.2}} \\
        Version control system.
    \item{\textbf{GitHub}} \\
        Remote code repository hosting, issue tracking, wiki for logging.
    \item{\textbf{MakerWare 2.3.1.18}} \\
        Compile 3D models.
    \item{\textbf{Google SketchUp 8.0.16846}} \\
        Create 3D models.
    \item{\textbf{Google Spreadsheets}} \\
        Keep lists organized.
    \item{\textbf{texlive}} \\
        Typesetting this report.
    \item{\textbf{python 2.7.4}} \\
        Writing the Galapagos assembler.
    \item{\textbf{Adobe Creative Cloud InDesign CC}} \\
        Designing the front page of this report.
    \item{\textbf{Lucidchart}} \\
        Creating charts.
\end{description}

\subsection{Hardware}
\begin{description}
\item{\textbf{Xilinx Spartan-6 XC6SLX45 CSG324}}
    FPGA board.
\item{\textbf{Energy Micro EFM32GG990F1024}}
    Microcontroller.
\item{\textbf{Energy Micro EFM32GG-DK3750}}
    Development kit used for testing code on the microcontroller before the PCB arrived.
\item{\textbf{Energy Micro EFM32GG-STK3700}}
    Prototyping and development when the microcontroller on the PCB could not be used.
\item{\textbf{Saleae Logic}}
    Logic analyzer
\item{\textbf{Development PC, Windows 7}}
    For development.
\item{\textbf{Mini USB cable}}
    For connecting the FPGA board to the development computer.
\item{\textbf{Makerbot Replicator 2}}
    3D printing the case.
\item{\textbf{Fluke Multimeter 77}}
    Checking currents on the PCB.
\item{\textbf{Altec Lansing ACS340}}
    Play sweet tunes in the lab while working.
\end{description}
