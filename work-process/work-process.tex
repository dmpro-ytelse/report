\section{Group Organization}

From early on, the group divided itself into 3 work groups, focusing on three main areas: FPGA, PCB and IO.
This was done largely based on the fact that it has been done similarly in the years before\cn, and it seemed like a reasonable thing to do.
One advantage in doing this is that each work group can become more specialized in their respective fields compared to if everyone were to work equally on everything.
This allows for a more advanced execution in the project, which is a good thing.
The group member work group allocation can be found in table \vref{table:group-allocation}.

Other than this group allocation, no other hierachical elements were introduced.
The group functioned as a direct democracy in all other issues, with no appointed leader.

\begin{table}[H]
\begin{center}
\begin{tabular}{l}
\textbf{FPGA} \\
\hline
Sigve Sebastian Farstad \\
Torbjørn Langland \\
Per Thomas Lundal \\
Bjørn Åge Tungesvik \\
\\
\\
\textbf{PCB} \\
\hline
Fedor Fadeev \\
Eirik Flogard \\
Rune Holmgren \\
Odd Magnus Trondrud \\
\\
\\
\textbf{IO} \\
\hline
Emil Taylor Bye \\
Péter Gombos \\
\end{tabular}
\caption{Group Allocation}
\label{table:group-allocation}
\end{center}
\end{table}

The group held weekly sync-up meetings in addition to a weekly status meeting with the course staff.
Unfortunately, meeting attendance was lower than what it should have been, due to poor discipline.
Fortunately, meeting minutes were always kept, so information from missed meetings did not go lost.

\section{Organizational tools}

\todo{github, trello, etc}
