\section{Development model}
For this project, it was agreed that an adaption of the Scrum development model should be used. The group wanted
to use an agile development method in order to be able to make quick changes to the requirement specification during the project.
The model is adapted however because working in sprints doesnt really fit for our use. We did not have daily meetings either, but rather
once in a week where everyone is informed about the current progress of the project.
\section{Group Organization}

From early on, the group divided itself into 3 work groups, focusing on three main areas: FPGA, PCB and IO.
This was done largely based on the fact that it has been done similarly in the years before\cn, and it seemed like a reasonable thing to do.
One advantage in doing this is that each work group can become more specialized in their respective fields compared to if everyone were to work equally on everything.
This allows for a more advanced execution in the project, which is a good thing.
The group member work group allocation can be found in table \vref{table:group-allocation}.

Other than this group allocation, no other hierachical elements were introduced.
The group functioned as a direct democracy in all other issues, with no appointed leader. This approach had both
it's benefits and downsides. The greatest benefit is that everyone got to take part in the important decisions made in the project.
The downside was that decisions also take much longer time to take due to having to discuss matters over a meeting.


\begin{table}[H]
\begin{center}
\begin{tabular}{l}
\textbf{FPGA} \\
\hline
Sigve Sebastian Farstad \\
Torbjørn Langland \\
Per Thomas Lundal \\
Bjørn Åge Tungesvik \\
\\
\\
\textbf{PCB} \\
\hline
Fedor Fadeev \\
Eirik Flogard \\
Rune Holmgren \\
Odd Magnus Trondrud \\
\\
\\
\textbf{IO} \\
\hline
Emil Taylor Bye \\
Péter Gombos \\
\end{tabular}
\caption{Group Allocation}
\label{table:group-allocation}
\end{center}
\end{table}

The group held weekly sync-up meetings in addition to a weekly status meeting with the course staff.
Unfortunately, meeting attendance was occationally some lower than what it should have been.
Fortunately, meeting minutes were always kept, so information from missed meetings did not go lost.

\section{Organizational tools}
\subsection{Github}
Github was used in order for everyone to be able to work at different parts of the project at the same time.
It also provides an excellent version control that would allow a user to work on experimental "branches" without
worrying about taking backups. These branches can also be merged into the main project later.
\subsection{trello}
Trello is a tool that the group used as "scrum table" to keep everyone updated in real time about the current progress in the project.
The tool resembles a scrum board which keeps track of what everyone is doing at a specific time.
\todo{github, trello, etc}

\section{Tools}

\todo{a huge list of tools that we've used: compilers, IDEs, microscopes, 3d-printers, logic analyzers, software, websites, calculators. If we've used it, it's going in here, and it's going to be detailed.}

\todo{The following is more or less yanxed from dmkons, so we need to go over it and make sure that the numbers are all correct.}

\subsection{Software}
\begin{description}
    \item{\textbf{ISE Project Navigator 12.4 (nt64) M.81d, expired licence}} \\
        Main IDE for writing VHDL.
    \item{\textbf{ISim 12.4 (nt64) M.81d, expired license}} \\
        Main simulation environment for simulating VHDL.
    \item{\textbf{ModelSim SE 6.6d}} \\
        Secondary simulation environment for simulating VHDL.
    \item{\textbf{Xilinx Platform Studio 12.4 (nt64) Build EDK\_MS4.81d+1, expired licence}} \\
        Used for preparing compiled VHDL for the FPGA board.
    \item{\textbf{Avnet Programming Utility}} \\
        Used for configuring the FPGA.
    \item{\textbf{Text editors}} \\
        Sublime Text 2, Vim 7.3, Notepad (©Copyright Microsoft Corporation).
    \item{\textbf{GNU command-line tools}} \\
        Grep, sed, find, etc.
    \item{\textbf{git 1.8.1.2}} \\
        Version control system.
    \item{\textbf{GitHub}} \\
        Remote code repository hosting, issue tracking, wiki for logging.
\end{description}

\subsection{Hardware}
\begin{description}
\item{\textbf{Xilinx Spartan-6 XC6SLX45 CSG324}}
    FPGA board.
\item{\textbf{Development PC, Windows 7}}
    For development.
\item{\textbf{Mini USB cable}}
    For connecting the FPGA board to the development computer.
\end{description}
