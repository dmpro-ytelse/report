
The functional requirements for the project are listed in table \vref{table:functional-requirements}. The non-functional requirements for the project are listed in table \vref{table:non-functional-requirements}.


Something about the product goals here.


 \subsection{Functional Requirements}

 \begin{table}
 \begin{center}
 \begin{tabular}{| l | l |}
 \hline
 Deliverable & Description \\
 \hline
 FR1 & The system should be a general computer.\\
 FR2 & The system should be able to solve hard problems using genetic alogorithms.\\
 FR3 & The system should be able to recieve instructions and data from an external entity.\\
 FR4 & The system should be able to send data to an external entity.\\
 FR5 & The product should have maximized performance rather than energy-efficiency. \\
 \hline
 \end{tabular}
 \caption{Functional Requirements}
 \label{table:functional-requirements}
 \end{center}
 \end{table}

 Something about the functional reqirements here.

 \subsection{Non-functional Requirements}

 \begin{table}
 \begin{center}
 \begin{tabular}{| l | l |}
 \hline
 Deliverable & Description \\
 \hline
 NFR1 & The system should be a MIMD computer.\\
 NFR2 & The system should be performant.\\
 PP3 & The instruction set should contain custom made instructions for faster execution of genetic algorithms. \\
 NFR3 & The system should have a 3D-printed case.\\
 NFR4 & The system should be possible to solder by hand, or use approved/known bgn packages\todo{which?}.\\
 NFR5 & The system should use a Spartan 6 FPGA.\\
 NFR6 & The system should use an Energy Micro microcontroller .\\
 NFR7 & The system should not cost more than 10000 NOK.\\
 NFR8 & A working demo program running a genetic algorithm.\\
 NFR9 & A report detailing the product and process.\\
 \hline
 \end{tabular}
 \caption{Non-functional Requirements}
 \label{table:non-functional-requirements}
 \end{center}
 \end{table}

Something about the non-functional requirements here.
