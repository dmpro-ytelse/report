
\section{Structure of this Report}

\section{Assignment}

\subsection{Assignment Text\TODO{TODO: cite assignment text}}

\begin{quote}

\begin{center}
TDT4295 Computer Design Project

Assignment Text

2013 
\end{center}
 
\textbf{Task: Construct a Multiple Program, Multiple Data}

The performance increase available from harvesting Instruction Level Parallelism (ILP) from the serial instruction stream is limited because we have reached the maximum power consumption that can be handled without expensive cooling solutions.
Consequently, there is a significant interest in single-chip parallel processor solutions.
The processor cores in commercial multi-core chips are conventional designs and therefore reasonably complex.
In this work, your task is to design a multiprocessor,Multiple Instruction, Multiple Data streams (MIMD).
A processor classified MIMD include a multiple-instruction stream organizations.
Such processors can executing different instructions, i.e. minimum 2 independent programs, on different (independent) data.

The task also include that a suitable application is chosen to demonstrate the processor.
Your processor will be implemented on an FPGA, and you are free to choose how to realize your MIMD computer architecture.
The system should be shown to work with a suitable application.
Studying the architecture of the Cell processor, or in general multi-core processors, can be a good starting point.
And a final tip; Keep it simple, as simple as possible, but not simpler.

Due to a large number of students this year, we will divide the work into two independent projects: a) Performance and b) Energy efficiency.
The goal of group a) is to achieve maximum performance while group b) should try to balance performance and energy.
The reports from both groups should include an evaluation of prototype performance and energy consumption.

\textbf{Additional requirements}

The unit must utilize an Energymicro micro controller and a Xilinx FPGA.
The budget is 10.000 NOK, which must cover components and PCB production.
The unit design must adhere to the limits set by the course staff at any given time.
Deadlines are given in a separate time schedule.

\textbf{Evaluation}

The project is evaluated based on the project report and an oral presentation of the work as well as a prototype demonstration.
One grade will be given to the group as a whole, unless there are significant variations in the amount of effort put into the project. 

\end{quote}

\subsection{Interpretation}

\TODO{TODO: Write about what product we are making here (i.e. a genetic solver)}

\section{Requirements}

The functional requirements for the project are listed in table \vref{table:functional-requirements}. The non-functional requirements for the project are listed in table \vref{table:non-functional-requirements}.

 \subsection{Functional Requirements}

 \begin{table}
 \begin{center}
 \begin{tabular}{| l | l | l |}
 \hline
 Functional Requirement & Description & Priority\\
 \hline
 FR1 & Some requirement. & High \\
 FR2 & Some requirement. & Medium \\
 \hline
 \end{tabular}
 \caption{Functional requirements.}
 \label{table:functional-requirements}
 \end{center}
 \end{table}

Something about the functional requirements here.

 \subsection{Non-functional Requirements}

 \begin{table}
 \begin{center}
 \begin{tabular}{| l | l | l |}
 \hline
 Non-functional Requirement & Description & Priority \\
 \hline
 NFR1 & Some requirement. & High \\
 NFR2 & Some requirement. & Medium \\
 \hline
 \end{tabular}
 \caption{Non-functional requirements.}
 \label{table:non-functional-requirements}
 \end{center}
 \end{table}

Something about the non-functional requirements here.
