This chapter aims to put this report in context.

\section{MIMD Computing}

\Gls{barricelli}, the solution computer presented in this report, is a \Gls{MIMD} computer. \Gls{MIMD}, or Multiple Instruction, Multiple Data, is a class of computer architectures involving multiple autonomous computing units executing different instructions on different data.
MIMD lends itself quite well to high performance computing through parallelism.

\todo{write more about MIMD}

\todo{use this image http://en.wikipedia.org/wiki/File:MIMD.svg for illustration purposed (remember to atttribute)}

\section{Genetic Algorithms}

A genetic algorithm is a search heuristic that is very useful for finding approximate solutions to hard optimization and search problems.
It mimics the natural process of evolution to find fit approximations in the search space.

A genetic algorithm represents possible solutions to a given problem as individuals of a population.
The individuals' fitness is calculated by a problem-specific fitness function, and fit individuals are combined together to create new individuals, mimicing the process of natural selection.
The idea is that after simulating a number of virtual generations of individuals, the fittest individual will have survived, and provides a good approximation to the solution.

\subsection{Concepts and Definitions}

To fully understand the inner workings of the \Gls{barricelli}, it is useful to become aquainted with the basic concepts of genetic algorithms.
This subsection of the document defines some genetic algorithm concepts that are used elsewhere in the report.

\subsubsection{Things\todo{better word for this}}

\paragraph{Individual}

An individual is any possible solution to a given problem.

\paragraph{Population}

A population is the group of all current individuals in a given experiment.

\paragraph{Search Space}

The search space is the set of all possible solutions to a given problem.

\paragraph{Chromosome}

A chromosome is a blueprint for a individual.

\paragraph{Trait}

A trait is a possible aspect of an individual.

\paragraph{Allele}

An allele is a possible setting for a trait.

\paragraph{Locus}

A locus is the position of a gene on a chromosome.

\paragraph{Genome}

A genome the collection of all chromosomes for an individual.

\subsubsection{Genetic operators}

\paragraph{Selection}

\paragraph{Crossover}

\paragraph{Mutation}

\subsection{Pseudo-code for a General Genetic Algorithm}

Algorithm \vref{algorithm:pseudo-code-for-ga} shows example pseudo code for a general genetic algorithm.

\begin{algorithm}[H]
\SetAlgoLined
\DontPrintSemicolon
\KwData{A population of random individuals}
\KwResult{A quite fit individual}
\Begin{
    $ k \longleftarrow 0 $\;
    $ P_k \longleftarrow $ a population of $ n $ randomly-generated individuals\;
    Compute $fitness(i)$ for each $ i \in P_k $\;
    \While{$ max(fitness(i) \in P_k) < threshold and k < maxIterations $}{
        Selection:\;
        Select $ (1 - x_i) \times n $ members of $ P_k $ and insert them into $ P_{k + 1} $\;

        Crossover:\;
        Select $ x_i \times n $ members of $ P_k $, pair them up, produce offspring, insert offspring into $ P_k $\;


        Mutation:\;
        Mutate $ \mu \times n $ members of $ P_{k + 1} $\;
        Compute $ fitness(i) $ for each $ i \in P_k $\;
        $k \longleftarrow k + 1 $
    }
    \Return{the fittest individual in $ P_k $}
}
\caption{Generic genetic algorithm}
\label{algorithm:pseudo-code-for-ga}
\end{algorithm}

\todo{tidy}

\subsection{Generational Genetic Algorithms}

Traditionally, genetic algorithms are implemented generationally \TODO{TODO: cite hromcovic or something}.
That means that an entire population of individuals are rated and combined to create a completely new population, and the old population is discarded.

\subsection{Continuous Genetic Algorithms}

A new and better strategy is gaining traction in the literature.\TODO{TODO: find a citation for this}.
Continuous genetic algorithms do away with the concept of discrete generations.
Rather, it continuously selects a few individuals, processes them, and returns them (or their offspring) to the population.
This erases the generational border, and is shown \TODO{TODO:cite our own research here, plus literature} to converge faster towards an optimal solution.

The genetic algorithm used in the Barricelli computer is a continuous genetic algorithm.


